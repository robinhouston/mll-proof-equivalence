\documentclass{robinminion}

% \usepackage[round]{natbib}
% \bibliographystyle{plainnat}

\usepackage{amsthm}

\author{Robin Houston}
\title{Everything I know about Subset Sum Reconfiguration}

\newtheorem{prop}{Proposition}
\newtheorem{eg}[prop]{Example}
\newtheorem{rem}[prop]{Remark}

\begin{document}
\maketitle
\linespread{1.1}\selectfont

\noindent This is a brief summary of everything I know about the subset sum reconfiguration problem as of four o'clock in the afternoon of Monday the 15th of July. It is brief by necessity, because I know so little.
%
\textbf{Update on Tuesday}: I found out another thing.

\begin{prop}
    Subset sum reconfiguration is weakly NP-hard. \textup(Demaine--Ito\textup)
\end{prop}
\begin{proof}
    The ordinary subset sum problem may be reduced to subset sum reconfiguration. Given elements
    $a_1, \dots, a_m$ with target sum $t$, build an SSR instance with elements $a_1, \dots, a_m$ having the corresponding weights,
    $t_1$ and $t_2$ having weight $t$, threshold values $k=t$ and $c=2t$, starting configuration $t_1$ and ending configuration
    $t_2$. At some point element $t_1$ must move from inside to outside: the first time it does, the
    inside set during the move must consist of some collection of $a$ elements that sum to $t$.
\end{proof}

\begin{prop}\label{prop:diameter}
    For all $n>1$, there is an $n$-element SSR instance whose configuration graph has diameter $2n-2$.
\end{prop}
\begin{proof}
    We exhibit in terms of a parameter $m$:
    \begin{itemize}
        \item an instance with $2m+2$ elements and a shortest solution with $4m+2$ steps;
        \item if $m>0$, an instance with $2m+1$ elements and a shortest solution with $4m+1$ steps.
    \end{itemize}
    The instance with $2m+2$ elements has elements $a_1, \dots, a_m$, $b_1, \dots, b_m$, $x$ and $y$.
    For the weights, let $w(a_i)=m+1$ and $w(b_i)=m+2$ for all $i$. Let $w(x)=w(y)=m(m+2)$. The
    starting configuration is $a_1, \dots, a_m, x$ and the target configuration is $a_1, \dots, a_m, y$.
    For threshold values let $k=m(m+2)$ and $c=2m(m+2)$.
    
    The first time $x$ moves, the weights of the remaining elements inside must sum to $m(m+2)$.
    In particular, these weights must sum to $m$ modulo $m+1$, hence the inside elements at this
    point must include all the $b$'s. Since the weight of $x$ plus the $b$'s equals the capacity $c$,
    the elements inside at this point must therefore be precisely all the $b$'s.
    %
    So all the $a$'s and $b$'s have to move twice, and $x$ and $y$ once each, for a total of
    $4m+2$ moves.
    
    To obtain the instance with $2m+1$ elements, remove the element $a_1$.
\end{proof}

\begin{rem}
    The lower bound exhibited in Proposition~\ref{prop:diameter} is optimal for $n<8$. I have verified this by exhaustive enumeration using the program \texttt{ssr\_graphs.py}.
\end{rem}

\begin{prop}\label{prop:element-moves-a-lot}
    For all $m$, there is a $(5m+3)$-element SSR instance that has an element that must move at least $2m+1$ times in any solution sequence.
\end{prop}
\begin{proof}
    The elements are $x_1, \dots, x_{2m+1}$ of weight $3$, $a_1, \dots, a_{3m+1}$ of weight $2$, and $b$ of weight $1$.
    Let $k=6m$ and $c=6m+3$. The starting configuration is $x_1, \dots, x_{2m+1}$, and the target is $a_1, \dots, a_{3m+1}, b$. For $i=0, 1, \dots, 2m$, consider the first move after which there are precisely $2m-i$ of the $x$ elements inside. That means the $a$'s and possibly $b$ inside at this point must sum to $3i$. If $i$ is odd then, since the $a$'s all have even weight, $b$ must be inside at this point; similarly if $i$ is even then $b$ must be outside. So element $b$ moves at least $2m+1$ times.
\end{proof}

\section*{New finding}
\noindent The construction of Proposition~\ref{prop:element-moves-a-lot} can be modified to give a quadratic lower bound for the diameter of the reconfiguration graph.
\begin{prop}
    For all $m$, there is a $(6m+2)$-element SSR instance that requires $2m^2+6m+2$ moves.
\end{prop}
\begin{proof}
    The elements are
    $x_1, \dots, x_{2m+1}$ of weight $3m$,
    $a_1, \dots, a_{3m+1}$ of weight $2m$,
    and $b_1, \dots, b_m$ of weight $1$.
    %
    Let $k=6m^2$ and $c=6m^2+3m$.
    %
    The starting configuration is $x_1, \dots, x_{2m+1}$,
    and the target is $a_1, \dots, a_{3m+1}, b_1, \dots, b_m$.
    %
    For $i=0, 1, \dots, 2m$, consider the first move after which there are precisely $2m-i$ of the $x$ elements inside. That means the $a$'s and $b$'s inside at this point must sum to $3im$. If $i$ is odd then, since the $a$'s all weigh $2m$, all the $b$'s must be inside at this point; similarly if $i$ is even then the $b$'s must all be outside. So the $b$ elements each move at least $2m+1$ times.
    %
    Therefore the total number of moves must be at least $(2m+1) + (3m+1) + (2m+1)m$, which is equal to $2m^2+6m+2$ as required.
\end{proof}
\end{document}
