\documentclass{robinminion}

% \usepackage[round]{natbib}
% \bibliographystyle{plainnat}

\usepackage{amsthm}

\author{Robin Houston}
\title{Notes on Subset Sum Reconfiguration}

\newtheorem{prop}{Proposition}
\newtheorem{lemma}[prop]{Lemma}
\newtheorem{conjecture}[prop]{Conjecture}
\newtheorem{eg}[prop]{Example}
\newtheorem{rem}[prop]{Remark}
\newtheorem*{definition}{Definition}

\let\union\cup
\let\intersection\cap

\begin{document}
\maketitle
\linespread{1.1}\selectfont

\noindent Some elementary observations about the subset sum reconfiguration problem

\section*{Examples}

\noindent The first example shows that some elements must sometimes move many times
\begin{eg}\label{eg:element-moves-a-lot}
    For all natural numbers $m$, there is a $(5m+3)$-element instance that has an element that must move at least $2m+1$ times in any solution sequence.
\end{eg}
\begin{proof}
    The elements are
    $x_1, \dots, x_{2m+1}$ of weight $3$,
    $a_1, \dots, a_{3m+1}$ of weight $2$,
    and $b$ of weight $1$.
    %
    Let $k=6m$ and $c=6m+3$.
    %
    The initial packing is $x_1, \dots, x_{2m+1}$,
    and the target is $a_1, \dots, a_{3m+1}, b$.
    %
    %
    At each packing in the sequence, count the number of $x$ elements it contains. The initial packing has $2m+1$ of them, and the target packing has none: therefore there must be packings in the sequence, in order, that contain $2m, 2m-1, \dots, 0$ of the $x$ elements. For $i=0, 1, \dots, 2m$, consider the first packing that contains precisely $2m-i$ of the $x$ elements. This packing must differ from its predecessor by the removal of some $x$ element, and so its total weight is forced to be $6m$. Hence the weights of the other elements in this packing -- $a$'s and possibly $b$ -- must sum to $3i$. If $i$ is odd then, since the $a$'s all have even weight, $b$ must be inside at this point; similarly if $i$ is even then $b$ must be outside. So element $b$ moves at least $2m+1$ times.
\end{proof}

\noindent The second example is similar, and establishes a quadratic lower bound for the diameter of the configuration graph.
\begin{eg}\label{eg:quadratic-bound}
    For all $m$, there is a $(6m+2)$-element instance that requires $2m^2+6m+2$ moves.
\end{eg}
\begin{proof}
    This example is similar to the previous one, except that all the weights are multiplied by $m$ and
    then the element $b$ is decomposed into $m$ elements of weight $1$. So the elements are
    $x_1, \dots, x_{2m+1}$ of weight $3m$,
    $a_1, \dots, a_{3m+1}$ of weight $2m$,
    and $b_1, \dots, b_m$ of weight $1$.
    %
    Let $k=6m^2$ and $c=6m^2+3m$.
    %
    The initial packing is $x_1, \dots, x_{2m+1}$,
    and the target is $a_1, \dots, a_{3m+1}, b_1, \dots, b_m$.
    %
    The argument is as before: for $i=0, 1, \dots, 2m$, consider the first packing that contains precisely $2m-i$ of the $x$ elements. The total weight of the non-$x$ elements in this packing must equal $3im$. If $i$ is odd then, since the $a$'s all weigh $2m$, all the $b$'s must be inside this packing; similarly if $i$ is even then the $b$'s must all be outside it. So the $b$ elements each move at least $2m+1$ times.
    %
    Therefore the total number of moves must be at least $(2m+1) + (3m+1) + (2m+1)m$, which is equal to $2m^2+6m+2$ as required.
\end{proof}

\section*{Eliminating redundant elements}

\noindent Elements that are `large enough' or `small enough' in weight may be eliminated, in the sense that we
can construct an equi-solvable instance by removing them and adjusting the threshold and capacity appropriately.
For example, the $b$ elements of Example~\ref{eg:quadratic-bound} are small enough to be eliminated -- and, once
they are eliminated, these instances admit linear-length solutions.

\subsection*{Eliminating large elements}

\noindent This part is really trivial. An element whose weight exceeds $c-k$ is too large to be moved in or out
of the packing, so it can take no part in any solution. Let us call such an element \emph{large}. If the instance
requires a large element to move, because it is contained in the initial packing but not the target or vice versa,
then it is clearly unsolvable. Otherwise there are two cases:
\begin{enumerate}
    \item The large element $x$ belongs to the initial and target packings. Remove $x$ from the set of elements and
    from the initial and target packings. If $w(x) <= k$ then reduce both $k$ and $c$ by $w(x)$. If $w(x) > k$ then
    let $k=0$ and reduce $c$ by $w(x)$.
    \item The large element $x$ belongs to neither the initial nor the target packing. Simple remove $x$ from the set of elements.
\end{enumerate}

\subsection*{Eliminating small elements}

\noindent The situation with small elements is rather the opposite. Of course small elements are able to move:
they have so much freedom of movement that they may be moved whenever necessary, so it is not necessary to
record them. In this section we assume that large elements have already been eliminated.

\begin{definition}\label{def:small}
    An element $x$ is \emph{small} if $w(x) < c-k-w_{\mathrm{max}} + 2$, where $w_{\mathrm{max}}$ is the
    maximum weight over all elements in the instance. In particular an element of weight $1$ is always small.
\end{definition}

\noindent To eliminate a small element, remove it from the set of elements and from the initial and target packings,
and adjust the threshold and capacity as follows:
\begin{enumerate}
    \item If $a$ belongs to the initial packing, leave $k$ unchanged and increase $c$ by $w(a)$.
    \item If $a$ does not belong to the initial packing and $k \geq w(a)$, leave $c$ unchanged and
            reduce $k$ by $w(a)$.
    \item If $a$ does not belong to the initial packing and $k < w(a)$, leave $c$ unchanged and set $k$ to zero.
\end{enumerate}



\begin{rem}
    This process increases $c-k$, which means that elements may be small in the reduced
    instance that were not small in the original. Therefore in some cases the construction may be iterated.
    For example we may eliminate an element of weight $1$, which then makes it possible to eliminate an
    element of weight $2$, etc.
\end{rem}

\end{document}
