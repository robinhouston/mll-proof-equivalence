\section{Proof nets and constraint graphs}


% TO DO: I think we can move away from partial configurations - this requires to show that edge-gadgets always belong to the kingdom of at least one vertex-gadget



\begin{definition}
A \emph{non-deterministic constraint graph} (\textsc{ncg}) $G=(V,E,c,v,w)$ consists of a set $V$ of vertices with \emph{minimum inflow constraint} $c\colon V\to\mathbb N$, and a set $E$ of at most one undirected edge $e$ per vertex-pair $v(e)=\{v_1,v_2\}$, with \emph{weight} $w\colon E\to N$.

A \emph{configuration} of an \textsc{ncg} is a partial function $\gamma\colon E\rightharpoonup V$ such that
\begin{itemize}
	\item
for every edge $e$, if $\gamma(e)$ is defined then $\gamma(e)\in v(e)$, and
	\item
for every vertex $v$, the sum of its inflow weights is at least its inflow constraint, $\sum\{w(e)\mid \gamma(e)=v\}\geq c(v)$.
\end{itemize}

A \emph{reconfiguration step} $\gamma\sim\delta$ connects two configurations for an \textsc{ncg} $G$ that differ in the assignment of exactly one edge.

\end{definition}



[[ NOTE: decide on notation for steps and paths in permutation relations ]]

For a graph $G=(V,E,c,v,w)$ let $|V|$ and $|E|$ denote the number of vertices and edges, respectively, and let $|c|$ and $|w|$ denote the sum of all inflow constraints, $\sum_{v\in V}c(v)$, and the sum of all edge weights, $\sum_{e\in E}w(e)$.


Let $A^n$ denote the sequent of $n$ copies of a formula $A$, and for a sequent $\Gamma=A_1,\dotsc,A_n$ let $\bigotimes\Gamma=A_1\tn\dotso\tn A_n$ and $\bigparr\Gamma=A_1\parr\dotso\parr A_n$.



\begin{definition}
\label{def:graph interpretation}
The \emph{interpretation} $\itn G$ of an \textsc{ncg} $G=(V,E,c,v,w)$ is a sequent constructed as follows.
%
Let $V=\{v_1,\dotsc,v_n\}$ and $E=\{e_1,\dotsc,e_m\}$ where $|V|=n$ and $|E|=m$.

The interpretation of a vertex $v_k$ is the formula
\[
	\itn{v_k} = \bigparr\big(C^{m\times c(v_k)} \big)\parr 1
\]
where each \emph{constraint element} $C$ is the formula
\[
	C = \bigparr\big(1^{3k+2}\big) \tn \bigparr\big(1^{3(n-k)+3}\big)
\]

The interpretation of an edge $e$ connecting vertices $v_i$ and $v_j$ with $i<j$ is the formula
\[
	\itn{e} = \bigotimes\big(W^{m\times w(e)}\big)\tn\bot
\]
where each \emph{weight element} $W$ is the formula
\[
	W = \bigotimes\big(\bot^{3i+2}\big)\parr\bigotimes\big(\bot^{3(j-i)+1}\big)\parr\bigotimes\big(\bot^{3(n-j)+3}\big)
\]

The interpretation of the graph $G$ is the sequent
\[
	%\itn G = \bigotimes_{1\leq i\leq n}\itn{v_i}, \itn{e_1},\dotsc,\itn{e_k}, \bot^m
	\itn G = \itn{v_1}\tn\dotso\tn\itn{v_n}, \itn{e_1},\dotsc,\itn{e_m}, 1^p
\]
where $p=m\times(|w|-|c|)\times(3n+4)$; the $p$ instances of $1$ are called \emph{edge-absorbers}.

\end{definition}



In an \textsc{ncg} $G$, a vertex $v$ and an edge $e$ will be called \emph{appropriate} (for each other) if $v\in v(e)$, and \emph{inappropriate} otherwise.
%
This notion is extended to vertex-gadgets $\itn v$ and edge-gadgets $\itn e$ in $\int G$.



For a weight element $W$ of an edge connecting $v_i$ and $v_j$, let the $\bot$-occurrences be named as follows,
\[
	W =  \big(\named{    \dagger}\bot\tn\named{   1}\bot\tn\dotsc\tn\named{3i+1}\bot\big)
	\parr\big(\named{   \ddagger}\bot\tn\named{3i+2}\bot\tn\dotsc\tn\named{3j+1}\bot\big)
	\parr\big(\named{\downdagger}\bot\tn\named{3j+2}\bot\tn\dotsc\tn\named{3n+3}\bot\big)
\]
and the $1$-occurrences of a constraint element $C$ in $\itn{v_k}$,
\[
	C = \big(\named{    \dagger'}1\parr\named{   1'}1\parr\dotso\parr\named{3k+1'}1\big)
	\tn \big(\named{\downdagger'}1\parr\named{3k+2'}1\parr\dotso\parr\named{3n+3'}1\big)~.
\]
There is a \emph{natural linking} for the sequent $W,C$ if $e$ is appropriate for $v_k$, i.e.\ if $k=i$ or $k=j$, as follows:
\[
	\links(x) = x' \qquad \text{for }x\in\{\dagger,\downdagger\}\cup\mathbb N
\]
\[
	\links(\ddagger)=
	\begin{cases}
			\dagger' & \text{if }k=i \\
		\downdagger' & \text{if }k=j~.
	\end{cases}
\]
There is also a natural linking for the sequent $W,\named\star 1,\named{1'}1,\dotsc,\named{3n+3'}1$ consisting of the weight element $W$ and $3n+4$ edge-absorbers:
\[
	\links(x)=
	\begin{cases}
		\star & \text{if }x\in\{\dagger,\ddagger,\downdagger\} \\
		   x' & \text{otherwise }.
	\end{cases}
\]


\begin{proposition}
\label{prop:element linkings}
\begin{enumerate}
	\item
Given a vertex $v$ and an appropriate edge $e$, for a constraint element $C$ in $\itn v$ and a weight element $W$ in $\itn e$ the natural linking for the sequent $W,C$ is a proof net.
	\item
Given a weight element $W$ for a graph with $|V|=n$, the natural linking for the sequent $W,1^{3n+4}$ is a proof net.
\end{enumerate}
\end{proposition}



The interpretation $\itn\gamma$ of a total configuration $\gamma$ for a graph $G$ is a linking $\links$ constructed incrementally, for each successive edge $e$, and for each successive weight element $W$ within $e$, as follows.
%
Let the rightmost $1$-occurrence of each vertex-gadget $\itn v$ be named $v$, and the rightmost $\bot$-occurrence of each edge-gadget $\itn e$ named $e$:
\[	
	\itn v = C\parr\dotso\parr C\parr \named v1 \qquad \itn e= W\tn\dotso\tn W\tn\named e\bot~.
\]
To define $\itn\gamma=\links$, firstly let
\[
	\links(e)=\gamma(e)~.
\]
Then successively for each edge $e$ and each weight element $W$ in $e$, if $\gamma(e)$ has a first free constraint element $C$, extend $\links$ to include the natural linking on $W,C$; otherwise, extend $\links$ by the natural linking on $W$ plus the first $3n+4$ free edge absorbers.


\begin{proposition}
If $\gamma$ is a total configuration for $G$ then $\itn\gamma$ is a proof net for $\itn G$.
\end{proposition}

\begin{proof}
Using Proposition~\ref{prop:scoping correctness}, it is sufficient to give a suitable scope for each $\parr$. 
%
The scope of each weight element $W$ is the sequent $W,C$ or $W,1^{3n+4}$ of its natural linking, which forms a proof net by Proposition~\ref{prop:element linkings}.
%
The scope of each vertex-gadget $\itn v$ contains the edge-gadgets $\itn e$ such that $\gamma(e)=v$, plus all the edge-absorbers within scopes of weight elements inside $\itn e$.
%
Since the weights of the connected edges $e$ sum to more than the inflow constraint of $v$, there are no unused constraint elements remaining in $\itn v$.
%
After contracting the scope of each $W$, each edge-gadget in the scope of $\itn v$ becomes a single string of connected vertices, connected to other edge-gadgets only via the special $\named v1$ of $\itn v$, thus forming a tree.
\end{proof}


%------------------------------------------------------------------------------------------------

\subsection{Completeness}

\begin{lemma}
If $\gamma\sim\delta$ for total configurations $\gamma$ and $\delta$, then $\itn\gamma\sim\itn\delta$.
\end{lemma}

%------------------------------------------------------------------------------------------------

\subsection{Soundness}


\begin{lemma}
In a proof net for $\itn G$, an edge-gadget $\itn e$ belongs to the empire of at most one vertex-gadget $\itn v$.
\end{lemma}

\begin{proof}
Since vertex-gadgets are joined by a tensor, the lemma is immediate from \cite[Proposition 1]{Bellin-vandeWiele-1995}.
\end{proof}




\begin{lemma}
\label{lem:appropriate edge weights}
In a proof net for $\itn G$, for each vertex $v$, the weights of the appropriate edge-gadgets in the empire of $\itn v$ are equal to or greater than the constraint of $v$.
\end{lemma}


\begin{proof}
Let $|V|=n$ and $|E|=k$, and consider the vertex $v_i$ and an edge $e$ connecting vertices $v_a$ and $v_b$ where $i\neq a,b$.
%
Each constraint element in $\itn{v_i}$ is an instance of the formula
\[
	C = \bigparr\big(1^{3i+2}\big) \tn \bigparr\big(1^{3(n-i)+3}\big)~.
\]
The two $\parr$-subformulae have a balance of $-3i-1$ and $-3(n-i)-2$ respectively.
%
Each weight element in $\itn{e}$ is an instance of 
\[
	W = \bigotimes\big(\bot^{3a+2}\big)\parr\bigotimes\big(\bot^{3(b-a)+1}\big)\parr\bigotimes\big(\bot^{3(n-b)+3}\big)~.
\]
The three $\tn$-subformulae have a net balance of $3a+1$, $3(b-a)$, and $3(n-b)+2$ respectively.
%
In pairs, they have a net balance of $3b+1$ (1st and 2nd subformula), $3(a+n-b)+3$ (1st and 3rd), and $3(n-a)+2$ (2nd and 3rd).
%
Since $i\neq a,b$, and since no $\tn$-subformula of $W$ can connect to more than one $\parr$-subformula of $C$, it follows that $W$ can balance the scope of at most one of both subformulae of $C$.


The vertex-gadget $\itn{v_i}$ is the formula
\[
	\itn{v_i} = \bigparr\big(C^{k\times c(v_i)} \big)\parr 1~.
\]
It will be shown that $k$ inappropriate edge-gadgets may balance at most $k-1$ constraint elements $C$.


Let the root nodes of the two $\parr$-subformulae of each instance of $C$ be labelled $x_j$ and $y_j$, for $1\leq j\leq k\times c(v_i)$. 
%
If the scopes $s(x_j)$ and $s(y_m)$ of any $x_j$ and $y_m$ are balanced by weight elements $W$ and $W'$ of the same edge-gadget $\itn e$, then since $W$ and $W'$ are connected by a tensor, there are switchings of $W$, $W'$, $s(x_j)$, and $s(y_m)$ such that $x_j$ and $y_m$ are connected in the proof net for $\itn G$.
%
Then the first constraint element of $\itn{v_i}$ requires 2 edge-gadgets to balance, and each successive element requires one additional edge-gadget.
\end{proof}



Using the above, a proof net for $\itn G$ may be interpreted as a configuration for $G$.



\begin{definition}
For a proof net $\links$ for the interpretation of a graph $\itn G$, let $\coitn\links$ be the configuration for $G$ where $\coitn\links(e)$ is $v$ if 1) $e$ is appropriate for $v$ and 2) $\itn e$ belongs to the empire of $\itn v$, and undefined otherwise.
\end{definition}





\begin{lemma}
If $\links\sim\links'$ are proof nets for $\itn G$ then $\coitn\links\sim\coitn{\links'}$.
\end{lemma}

\begin{proof}
A single permutation $\links\sim\links'$ on proof nets may move a number of edge-gadgets $\itn{e_1},\dotso,\itn{e_n}$ in three ways: 1) out of the empire of a vertex-gadget $\int v$, 2) into the empire of a vertex-gadget $\int w$, or 3) both.
%
Then in both $\coitn\links$ and $\coitn{\links'}$ the edges $e_1$ through $e_n$ are mobile, since by Lemma~\ref{lem:appropriate edge weights} the empires of the vertex-gadgets $\itn v$ (in cases 1 and 3) and $\itn w$ (in cases 2 and 3) contain appropriate edge-gadgets other than $\itn{e_1}$ through $\itn{e_n}$ of sufficient combined weight.
%
It follows that in the graph $G$ the edges $e_1$ through $e_n$ can be moved away from $v$ and/or onto $w$ one at a time.

\end{proof}




%\begin{definition}
%The interpretation $\itn\gamma$ of a configuration $\gamma$ for a graph $G$ is the set of proof nets for $\itn G$ satisfying the following constraint: if $\gamma(e)=v$ then $\itn e$ belongs to the empire of $\itn v$
%\end{definition}








