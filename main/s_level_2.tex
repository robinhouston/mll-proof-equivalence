\section{Proof nets and constraint graphs}


\newcommand\itn[1]{\llbracket#1\rrbracket}
\newcommand\coitn[1]{\llparenthesis#1\rrparenthesis}

\begin{definition}
A \emph{non-deterministic constraint graph} (\textsc{ncg}) $G=(V,E,c,v,w)$ consists of a set $V$ of vertices with \emph{minimum inflow constraint} $c\colon V\to\mathbb N$, and a set $E$ of at most one undirected edge $e$ per vertex-pair $v(e)=\{v_1,v_2\}$, with \emph{weight} $w\colon E\to N$.

A \emph{configuration} of an \textsc{ncg} is a function $\gamma\colon E\to V$ such that
\begin{itemize}
	\item
$\gamma(e)\in v(e)$ for every edge $e$, and
	\item
for every vertex $v$, the sum of its inflow weights $\sum\{w(e)\mid \gamma(e)=v\}$ is equal to or greater than its inflow constraint $c(v)$.
\end{itemize}

A \emph{reconfiguration step} $\gamma\sim\delta$ connects two configurations for a \textsc{ncg} $G$ that differ in the assignment $\gamma(e)$ of exactly one edge.

\end{definition}

[[ NOTE: decide on notation for steps and paths in permutation relations ]]

For a graph $G=(V,E,c,w)$ let $|V|$ and $|E|$ denote the number of vertices and edges, respectively, and let $|c|$ and $|w|$ denote the sum of all inflow constraints, $\sum_{v\in V}c(v)$, and the sum of all edge weights, $\sum_{e\in E}w(e)$.


Let $A^n$ denote the sequent of $n$ copies of a formula $A$, and for a sequent $\Gamma=A_1,\dotsc,A_n$ let $\bigotimes\Gamma=A_1\tn\dotso\tn A_n$ and $\bigparr\Gamma=A_1\parr\dotso\parr A_n$.



\begin{definition}
The \emph{interpretation} $\itn G$ of an \textsc{ncg} $G=(V,E,c,w)$ is a sequent constructed as follows.
%
Let $V=\{v_1,\dotsc,v_n\}$ and $E=\{e_1,\dotsc,e_k\}$ where $|V|=n$ and $|E|=k$.

The interpretation of a vertex $v_i$ is the formula
\[
	\itn{v_i} = \bigparr\big(C^{k\times c(v_i)} \big)\parr 1
\]
where each \emph{constraint element} $C$ is the formula
\[
	C = \bigparr\big(1^{3i+2}\big) \tn \bigparr\big(1^{3(n-i)+3}\big)
\]

The interpretation of an edge $e$ connecting vertices $v_i$ and $v_j$ with $i<j$ is the formula
\[
	\itn{e} = \bigotimes\big(W^{k\times w(e)}\big)\tn\bot
\]
where each \emph{weight element} $W$ is the formula
\[
	W = \bigotimes\big(\bot^{3i+2}\big)\parr\bigotimes\big(\bot^{3(j-i)+1}\big)\parr\bigotimes\big(\bot^{3(n-j)+3}\big)
\]

The interpretation of the graph $G$ is the sequent
\[
	%\itn G = \bigotimes_{1\leq i\leq n}\itn{v_i}, \itn{e_1},\dotsc,\itn{e_k}, \bot^m
	\itn G = \itn{v_1}\tn\dotso\tn\itn{v_n}, \itn{e_1},\dotsc,\itn{e_k}, \bot^m
\]
where $m=k\times(|w|-|c|)\times(3n+4)$.

\end{definition}



\begin{lemma}
In a proof net for $\itn G$, an edge-gadget $\itn e$ belongs to the empire of at most one vertex-gadget $\itn v$.
\end{lemma}

\begin{proof}
Since vertex-gadgets are joined by a tensor, the lemma is immediate from \cite[Proposition 1]{Bellin-vandeWiele-1995}.
\end{proof}



%\begin{lemma}
%If the permutation $\links\sim\links'$ consists of the re-attachment of exactly one jump, then $\coitn\links$ and $\coitn{\links'}$ differ on the value of at most one edge.
%\end{lemma}

% THIS IS FALSE : OTHER EDGES MAY BE CONNECTED *VIA* THE ONE THAT IS MOVED


By the above, a proof net for $\itn G$ gives a partial function from edges to vertices.

\begin{definition}
	For a proof net $\links$ for the interpretation of a graph $\itn G$, let $\coitn\links\colon E\rightharpoonup V$ be the partial function where $\coitn\links(e)$ is $v$ if $\itn e$ belongs to the empire of $\itn v$, and undefined otherwise.
\end{definition}

In an \textsc{ncg} $G$, an edge $e$ will be called \emph{appropriate} for a vertex $v$ if $v\in v(e)$, and \emph{inappropriate} otherwise.
%
This notion is extended to edge-gadgets $\itn e$ and vertex-gadgets $\itn v$ in $\int G$.


\begin{lemma}
If the permutation $\links\sim\links'$ consists of the re-attachment of exactly one jump, then from $\coitn\links$ to $\coitn{\links'}$ at most one edge is taken from one appropriate vertex to another.
\end{lemma}

\begin{proof}[Proof sketch]
At most, a link can re-connect from vertex-gadget $v$ to vertex-gadget $w$, moving a number of chained edge-gadgets from $v$ to $w$.
%
But at most one edge is appropriate for both $v$ and $w$.
\end{proof}




\begin{lemma}
In a proof net for $\itn G$, for each vertex $v$, the weights of the appropriate edge-gadgets in the empire of $\itn v$ are equal to or greater than the constraint of $v$.
\end{lemma}



\begin{proof}[Proof sketch]
There are not enough edges to fill a unit of weight inappropriately in $\itn v$.
\end{proof}








\begin{definition}
The interpretation $\itn\gamma$ of a configuration $\gamma$ for a graph $G$ is the set of proof nets for $\itn G$ satisfying the following constraint: if $\gamma(e)=v$ then $\itn e$ belongs to the empire of $\itn v$
\end{definition}








