\section{Proof nets and constraint graphs}



\newcommand\alt{\underline}

[[ NOTE: we should decide on notation for steps versus paths in permutation relations ]]



\begin{definition} 
A \emph{non-deterministic constraint graph} (\textsc{ncg}) $G=(V,E,c,v,w)$ consists of a set $V$ of vertices with \emph{minimum inflow constraint} $c\colon V\to\mathbb N$, and a set $E$ of at most one undirected edge $e$ per vertex-pair $v(e)=\{v_1,v_2\}$, with \emph{weight} $w\colon E\to N$.

A (partial) \emph{configuration} of an \textsc{ncg} is a (partial) function $\gamma\colon E\to V$ such that
\begin{itemize}
	\item
for every edge $e$, $\gamma(e)\in v(e)$ or (in the partial case) $\gamma(e)$ is undefined, and
	\item
for every vertex $v$, the sum of its inflow weights is at least its inflow constraint, $\sum\{w(e)\mid \gamma(e)=v\}\geq c(v)$.
\end{itemize} 

A \emph{reconfiguration step} $\gamma\sim\delta$ connects two (partial) configurations for an \textsc{ncg} $G$ that differ in the assignment of exactly one edge.

\end{definition}


The \emph{(partial) \textsc{ncg}-reconfiguration problem} is the problem of deciding when two (partial) configurations are connected by a path in the graph of (partial) configurations and reconfiguration steps.

\begin{theorem}[\cite{demaine}]
\textsc{ncg}-reconfiguration is \textsc{pspace}-complete.
\end{theorem}

\begin{proposition}
Partial \textsc{ncg}-reconfiguration is \textsc{pspace}-complete.
\end{proposition}

\begin{proof}
There is a path between total configurations $\gamma$ and $\delta$ in partial \textsc{ncg}-reconfiguration if and only if there is one in \textsc{ncg}-reconfiguration, for the following two reasons.
%
Firstly, if $\gamma\sim\delta$ are partial configurations, they may be completed to total configurations $\gamma'\sim\delta'$ or $\gamma'=\delta'$.
%
Secondly, if $\gamma'$ and $\gamma''$ are total configurations that both agree with a partial configuration $\gamma$ where it is defined, then $\gamma'$ and $\gamma''$ are connected in \textsc{ncg}-reconfiguration by re-assigning the values where they disagree.
%
\end{proof}


For a graph $G=(V,E,c,v,w)$ let $|V|$ and $|E|$ denote the number of vertices and edges, respectively, and let $|c|$ and $|w|$ denote the sum of all inflow constraints, $\sum_{v\in V}c(v)$, and the sum of all edge weights, $\sum_{e\in E}w(e)$.


Let $A^n$ denote the sequent of $n$ copies of a formula $A$, and for a sequent $\Gamma=A_1,\dotsc,A_n$ let $\bigotimes\Gamma=A_1\tn\dotso\tn A_n$ and $\bigparr\Gamma=A_1\parr\dotso\parr A_n$.



\begin{definition}
\label{def:graph interpretation}
The \emph{interpretation} $\itn G$ of an \textsc{ncg} $G=(V,E,c,v,w)$ is a sequent constructed as follows.
%
Let $V=\{v_1,\dotsc,v_n\}$ and $E=\{e_1,\dotsc,e_m\}$ where $|V|=n$ and $|E|=m$.

The interpretation of a vertex $v_k$ is the formula
\[
	\itn{v_k} = \bigparr\big(C^{m\times c(v_k)} \big)\parr 1
\]
where each \emph{constraint element} $C$ is the formula
\[
	C = \bigparr\big(1^{3k+2}\big) \tn \bigparr\big(1^{3(n-k)+3}\big)
\]

The interpretation of an edge $e$ connecting vertices $v_i$ and $v_j$ with $i<j$ is the formula
\[
	\itn{e} = \bigotimes\big(W^{m\times w(e)}\big)\tn\bot
\]
where each \emph{weight element} $W$ is the formula
\[
	W = \bigotimes\big(\bot^{3i+2}\big)\parr\bigotimes\big(\bot^{3(j-i)+1}\big)\parr\bigotimes\big(\bot^{3(n-j)+3}\big)
\]

The interpretation of the graph $G$ is the sequent
\[
	%\itn G = \bigotimes_{1\leq i\leq n}\itn{v_i}, \itn{e_1},\dotsc,\itn{e_k}, \bot^m
	\itn G = \itn{v_1}\tn\dotso\tn\itn{v_n}, \itn{e_1},\dotsc,\itn{e_m}, 1^p
\]
where $p=m\times(|w|-|c|)\times(3n+4)$; the $p$ instances of $1$ are called \emph{edge-absorbers}.

\end{definition}



In an \textsc{ncg} $G$, a vertex $v$ and an edge $e$ will be called \emph{appropriate} (for each other) if $v\in v(e)$, and \emph{inappropriate} otherwise.
%
This notion is extended to vertex-gadgets $\itn v$ and edge-gadgets $\itn e$ in $\itn G$.



For a weight element $W$ of an edge connecting $v_i$ and $v_j$, let the $\bot$-occurrences be named as follows,
\[
	W =  \big(\named{    \dagger}\bot\tn\named{   1}\bot\tn\dotsc\tn\named{3i+1}\bot\big)
	\parr\big(\named{   \ddagger}\bot\tn\named{3i+2}\bot\tn\dotsc\tn\named{3j+1}\bot\big)
	\parr\big(\named{\downdagger}\bot\tn\named{3j+2}\bot\tn\dotsc\tn\named{3n+3}\bot\big)
\]
and the $1$-occurrences of a constraint element $C$ in $\itn{v_k}$,
\[
	C = \big(\named{\alt    \dagger}1\parr\named{\alt{   1}}1\parr\dotso\parr\named{\alt{3k+1}}1\big)
	\tn \big(\named{\alt\downdagger}1\parr\named{\alt{3k+2}}1\parr\dotso\parr\named{\alt{3n+3}}1\big)~.
\]
There is a \emph{natural linking} for the sequent $W,C$ if $e$ is appropriate for $v_k$, i.e.\ if $k=i$ or $k=j$, as follows:
\[
	\links(x) = \alt x \qquad \text{for }x\in\{\dagger,\downdagger\}\cup\mathbb N
\]
\[
	\links(\ddagger)=
	\begin{cases}
		\alt	\dagger & \text{if }k=i \\
		\alt\downdagger & \text{if }k=j~.
	\end{cases}
\]
There is also a natural linking for the sequent $W,\named\star 1,\named{\alt 1}1,\dotsc,\named{\alt {3n+3}}1$ consisting of the weight element $W$ and $3n+4$ edge-absorbers:
\[
	\links(x)=
	\begin{cases}
		\star  & \text{if }x\in\{\dagger,\ddagger,\downdagger\} \\
		\alt x & \text{otherwise }.
	\end{cases}
\]


\begin{proposition}
\label{prop:element linkings}
\begin{enumerate}
	\item
Given a vertex $v$ and an appropriate edge $e$, for a constraint element $C$ in $\itn v$ and a weight element $W$ in $\itn e$ the natural linking for the sequent $W,C$ is a proof net.
	\item
Given a weight element $W$ for a graph with $|V|=n$, the natural linking for the sequent $W,1^{3n+4}$ is a proof net.
\end{enumerate}
\end{proposition}


The rightmost $1$-occurrence of each vertex-gadget $\itn v$, named $\alt v$ below, and the rightmost $\bot$-occurrence of each edge-gadget $\itn e$, named $\alt e$, will be called the \emph{indicator vertices} of $\itn v$ and $\itn e$.
\[	
	\itn v = C\parr\dotso\parr C\parr \named{\alt v}1 \qquad \itn e= W\tn\dotso\tn W\tn\named{\alt e}\bot~.
\]


\begin{definition}
\label{def:configuration interpretation}
The \emph{interpretation} $\itn\gamma$ of a total configuration $\gamma$ for a graph $G$ is a linking $\links$ constructed incrementally, for each successive edge $e$, and for each successive weight element $W$ within $e$, as follows.
%
Let $\gamma(e)=v$; firstly, the the indicator vertex of $\itn e$ links to the indicator of $\itn v$.
%
Then successively for each weight element $W$ in $e$, if $\itn v$ has a first free constraint element $C$, extend $\itn\gamma$ to include the natural linking on $W,C$; otherwise, extend $\itn\gamma$ by the natural linking on the sequent consisting of $W$ plus the first $3n+4$ free edge absorbers.
\end{definition}


\begin{proposition}
If $\gamma$ is a total configuration for $G$ then $\itn\gamma$ is a proof net for $\itn G$.
\end{proposition}

\begin{proof}
Using Proposition~\ref{prop:scoping correctness}, it is sufficient to give a suitable scope for each $\parr$. 
%
The scope of each weight element $W$ is the sequent $W,C$ or $W,1^{3n+4}$ of its natural linking, which forms a proof net by Proposition~\ref{prop:element linkings}.
%
The scope of each vertex-gadget $\itn v$ contains the edge-gadgets $\itn e$ such that $\gamma(e)=v$, plus all the edge-absorbers within scopes of weight elements inside $\itn e$.
%
Since the weights of the connected edges $e$ sum to more than the inflow constraint of $v$, there are no unused constraint elements remaining in $\itn v$.
%
After contracting the scope of each $W$, each edge-gadget in the scope of $\itn v$ becomes a single string of connected vertices, connected to other edge-gadgets only via the special $\named v1$ of $\itn v$, thus forming a tree.
\end{proof}


%------------------------------------------------------------------------------------------------

\subsection{Completeness}


In a proof net for $\itn G$, an edge-gadget $\itn e$ that is in the empire of an appropriate vertex-gadget $\itn v$ is \emph{naturally linked} if the indicator of $\itn e$ connects to the indicator of $\itn v$, and each weight element $W$ of $\itn e$ is either naturally linked to a constraint element $C$ in $\itn v$ or linked only to edge-absorbers.

\begin{lemma}
\label{lem:octopus roll}
In a proof net $\links$ for a sequent $\Gamma=\itn v,\itn{e_1},\dotsc,\itn{e_m},\bot^p$ where each edge-gadget is naturally linked, if a weight element $W_i$ in $\itn{e_i}$ is linked to $C$ in $\itn v$ and $W_j$ in $\itn{e_j}$ is linked to edge-absorbers $\bot^n$, then there is a net $\links'\sim\links$ in which $W_j$ is naturally linked to $C$, $W_i$ is linked to $\bot^n$, and $\links'$ agrees with $\links$ otherwise.
\end{lemma}

\begin{proof}
Let $W_i=A\parr B\parr C$, $W_j=D\parr E\parr F$, and $C=X\tn Y$.
%
We will illustrate the path of permutations for the case where $A,B,X$ and $D,X$, and thus also $C,Y$ and $E,F,Y$, are balanced sequents; other cases are similar.
%
\begin{enumerate}
	\item
The initial configuration is illustrated below; other weight and constraint elements are omitted, and $v$, $i$, and $j$ are the indicator vertices of $\itn v$, $\itn{e_i}$, and $\itn{e_j}$ respectively.
\[
	\octorollA1
\]
	\item
The link $i-v$ is re-attached to connect to an edge-absorber together with only the links from the first $\bot$ of each of $D$, $E$, and $F$. 
\[
	\octorollB2
\]
	\item
Secondly, the link from the first $\bot$ of $D$ is moved to $X$, and those of $E$ and $F$ are moved to $Y$.
\[
	\octorollB3
\]
	\item\label{item:exchange added}
There are subnets over the sub-sequents $A,B,X,D,\bot^m$ and $C,Y,E,F,\bot^k$.
%
These subnets may be rewired so that $D\parr E\parr F$ is naturally linked to $X\tn Y$: by Proposition~\ref{prop:parity determines equivalence}, the resulting subnets are equivalent as long as their parity is preserved.
%
Two links from $C$ to $X$ should remain exchanged, compared to the natural linking, for step \ref{item:exchange needed} below.
%
The links of $A,B,C$ connect to the edge-absorbers $\bot^m$ and $\bot^k$, with one remaining link from $B$ to $X$ and one from $C$ to $Y$.
\[
	\octorollC
\]
	\item
The link from $C$ to $Y$ is moved towards an edge-absorber connected to $A,B$.
\[
	\octorollD1
\]
	\item\label{item:exchange needed}
The link from $B$ to $X$ is the one remaining connection between the edge-gadgets $\itn{e_i}$ and $\itn{e_j}$.
%
Lemma~\ref{lem:double exchange} allows to swap the targets of the link from $B$ to $X$ and the link from $i$, and simultaneously undo the exchange in the links from $C$ to $X$ added in step \ref{item:exchange added} above.
\[
	\octorollD2
\]
	\item
The link from $i$ may be re-attached to $v$ to yield the final configuration.
\[
	\octorollA2
\]
\end{enumerate}
\end{proof}


An edge-gadget $\itn e$ is \emph{free} if each of its weight elements is linked only to edge-absorbers.


\begin{lemma}
If $\links$ is a naturally linked proof net for $\itn G$ with a free edge-gadget $\itn e$, and $\links'$ agrees with $\links$ up to an even permutation of edge-absorbers, then $\links\sim\links'$.
\end{lemma}

\begin{proof}
Let $e_0$ be the indicator vertex of $\itn e$; since $\itn e$ is free, $e_0$ may re-attach anywhere within the proof net.
%
Let $e_1$ and $e_2$ be arbitrary other $\bot$-occurrences in $\itn e$.

\begin{enumerate}

	\item
To permute two edge-absorbers $v$ and $w$ linked to by other edge-gadgets than $\itn e$, attach $e_0$ to $v$, and apply Lemma~\ref{lem:double exchange} to exchange $v$ and $w$, also exchanging the targets of $e_0$ and $e_1$.

	\item
To reinstate $e_0$ as the connection between $\itn e$ and the remainder of the proof net, either perform another permutation as above, or permute $v$ and $w$ twice again, once exchanging the targets of $e_1$ and $e_2$, and once exchanging the targets of $e_2$ and $e_0$.

	\item
To exchange an edge-absorber $w$ linked from a $\named d\bot$ outside $\itn e$ for the target $v$ of $e_1$, and attach $e_0$ to $w$, attach $d$ to $v$.
%
At this point, $e_1$ forms the only connect between $\itn e$ and the remainder of the proof net; to reinstate $e_0$ in this role, do as above, producing a net effect of cycling the targets of $e_1$, $e_2$, and $d$.

	\item
Finally, to permutate edge-absorbers $v$ and $w$ linked by $\itn e$, attach $e_0$ to the indicator of a vertex-gadget where also another edge-gadget $\itn d$ is connected, with indicator vertex $d_0$ and an arbitrary other vertex $d_1$.
%
Connect $d_0$ to $v$, and apply Lemma~\ref{lem:double exchange} to permute $v$ and $w$, as well as exchanging the targets of $d_0$ and $d_1$.
%
Perform a second permutation to return $d_0$ and $d_1$ their proper edge-absorbers.

\end{enumerate}

In each case, if one of the edge-absorbers exchanged is linked to by multiple $\bot$-occurrences within the same weight element, these may be termporarily attached elsewhere.

\end{proof}


\begin{lemma}
If $\gamma\sim\delta$ for total configurations $\gamma$ and $\delta$, then $\itn\gamma\sim\itn\delta$.
\end{lemma}

%------------------------------------------------------------------------------------------------

\subsection{Soundness}


\begin{lemma}
In a proof net for $\itn G$, an edge-gadget $\itn e$ belongs to the empire of at most one vertex-gadget $\itn v$.
\end{lemma}

\begin{proof}
Since vertex-gadgets are joined by a tensor, the lemma is immediate from \cite[Proposition 1]{Bellin-vandeWiele-1995}.
\end{proof}




\begin{lemma}
\label{lem:appropriate edge weights}
In a proof net for $\itn G$, for each vertex $v$, the weights of the appropriate edge-gadgets in the empire of $\itn v$ are equal to or greater than the constraint of $v$.
\end{lemma}


\begin{proof}
Let $|V|=n$ and $|E|=m$, and consider the vertex $v_i$ and an edge $e$ connecting vertices $v_a$ and $v_b$ where $i\neq a,b$.
%
Each constraint element in $\itn{v_i}$ is an instance of the formula
\[
	C = \bigparr\big(1^{3i+2}\big) \tn \bigparr\big(1^{3(n-i)+3}\big)~.
\]
The two $\parr$-subformulae have a balance of $-3i-1$ and $-3(n-i)-2$ respectively.
%
Each weight element in $\itn{e}$ is an instance of 
\[
	W = \bigotimes\big(\bot^{3a+2}\big)\parr\bigotimes\big(\bot^{3(b-a)+1}\big)\parr\bigotimes\big(\bot^{3(n-b)+3}\big)~.
\]
The three $\tn$-subformulae have a net balance of $3a+1$, $3(b-a)$, and $3(n-b)+2$ respectively.
%
In pairs, they have a net balance of $3b+1$ (1st and 2nd subformula), $3(a+n-b)+3$ (1st and 3rd), and $3(n-a)+2$ (2nd and 3rd).
%
Since $i\neq a,b$, and since no $\tn$-subformula of $W$ can connect to more than one $\parr$-subformula of $C$, it follows that $W$ can balance the scope of at most one of both subformulae of $C$.


The vertex-gadget $\itn{v_i}$ is the formula
\[
	\itn{v_i} = \bigparr\big(C^{m\times c(v_i)} \big)\parr 1~.
\]
It will be shown that $m$ inappropriate edge-gadgets may balance at most $m-1$ constraint elements $C$.


Let the root nodes of the two $\parr$-subformulae of each instance of $C$ be labelled $x_j$ and $y_j$, for $1\leq j\leq m\times c(v_i)$. 
%
If the scopes $s(x_j)$ and $s(y_m)$ of any $x_j$ and $y_m$ are balanced by weight elements $W$ and $W'$ of the same edge-gadget $\itn e$, then since $W$ and $W'$ are connected by a tensor, there are switchings of $W$, $W'$, $s(x_j)$, and $s(y_m)$ such that $x_j$ and $y_m$ are connected in the proof net for $\itn G$.
%
Then the first constraint element of $\itn{v_i}$ requires 2 edge-gadgets to balance, and each successive element requires one additional edge-gadget.
\end{proof}



Using the above, a proof net for $\itn G$ may be interpreted as a configuration for $G$.



\begin{definition}
For a proof net $\links$ for the interpretation of a graph $\itn G$, let $\coitn\links$ be the partial configuration for $G$ where $\coitn\links(e)$ is $v$ if 1) $e$ is appropriate for $v$ and 2) $\itn e$ belongs to the empire of $\itn v$, and undefined otherwise.
\end{definition}





\begin{lemma}
If $\links\sim\links'$ are proof nets for $\itn G$ then $\coitn\links\sim\coitn{\links'}$.
\end{lemma}

\begin{proof}
A single permutation $\links\sim\links'$ on proof nets may move a number of edge-gadgets $\itn{e_1},\dotso,\itn{e_n}$ in three ways: 1) out of the empire of a vertex-gadget $\int v$, 2) into the empire of a vertex-gadget $\int w$, or 3) both.
%
Then in both $\coitn\links$ and $\coitn{\links'}$ the edges $e_1$ through $e_n$ are mobile, since by Lemma~\ref{lem:appropriate edge weights} the empires of the vertex-gadgets $\itn v$ (in cases 1 and 3) and $\itn w$ (in cases 2 and 3) contain appropriate edge-gadgets other than $\itn{e_1}$ through $\itn{e_n}$ of sufficient combined weight.
%
It follows that in the graph $G$ the edges $e_1$ through $e_n$ can be moved away from $v$ and/or onto $w$ one at a time.

\end{proof}











