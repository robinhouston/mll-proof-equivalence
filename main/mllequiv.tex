\documentclass{robinminion}


\usepackage{amsthm}
\usepackage[round]{natbib}
\usepackage{mllequiv,willemtools}

\makeTheoremDefs


\author{Willem Heijltjes and Robin Houston}
\title{The proof equivalence problem for multiplicative linear logic is \textsc{pspace}-complete v0.3}


\begin{document}

\maketitle

\begin{abstract}
MLL proof equivalence is the problem of deciding whether two proofs are related by a series of rule permutations.
%
Previous work has shown the problem to be equivalent to a rewiring problem on proof nets, which are not canonical for full MLL due to the presence of the two units.
%
Drawing from recent work on reconfiguration problems, in this paper it is shown that MLL proof equivalence is PSPACE-complete, using a reduction from Nondeterministic Constraint Logic.
\end{abstract}



%\include{s_intro}



\section{MLL}



\begin{figure}
\[
\begin{array}{llll}
	\MLLrule b
 &	\MLLrule 1
 &	\MLLrule p
 &	\MLLrule t
\end{array}
\]
\caption{Inference rules for unit-only \MLL}
\label{fig:MLL}
\end{figure}



\begin{figure}
\[
	\vc{\MLLperm{bb1}} \perm \vc{\MLLperm{bb2}}
\qquad
	\vc{\MLLperm{bp1}} \perm \vc{\MLLperm{bp2}}
\]
\[
	\vc{\MLLperm{bt1}} \perm \vc{\MLLperm{bt2}} \perm \vc{\MLLperm{bt3}}
\]
\[
	\vc{\MLLperm{pp1}} \perm \vc{\MLLperm{pp2}}
\]
\[
	\vc{\MLLperm{pt1}} \perm \vc{\MLLperm{pt2}}
\]
\[
	\vc{\MLLperm{tt1}} \perm \vc{\MLLperm{tt2}}
\]
\caption{Permutations}
\label{fig:permutations}
\end{figure}



The formulae of unit-only multiplicative linear logic are given by the following grammar.
%
\[
	A,B,C \coloneq \bot \mid 1 \mid A\parr B \mid A\tn B
\]
%
The connectives $\tn$ and $\parr$ will be considered up to associativity, and \emph{duality} $\dual A$ is via DeMorgan.
%
A \emph{sequent} $\Gamma,\Delta$ will be a multiset of formulae.
%
Within a sequent, connectives and units will be \emph{named} with distinct elements from an arbitrary set of names $N$, e.g.\
$\named a1\named b\parr\named c1,\named d\bot\named e\tn\named f\bot$.
%
This allows to 1) avoid using the notion of \emph{occurrence}, and instead refer to subformulae by the name of their root connective, as e.g.\ $\named bA$, 2) distinguish the two proofs of the above sequent while using standard multiset sequents, and 3) easily extract proof nets, as graphs using the names of connectives as vertices.
%
Names will mostly be left implicit.



Proofs are constructed from the inference rules in Figure~\ref{fig:MLL}.
%
The names of connectives are preserved through inferences.
%
Only cut-free proofs are considered, and no cut-rule is added.
%
\emph{Permutations} of inference rules are displayed in Figure~\ref{fig:permutations}; the symmetric variants of the last two permutations, \emph{par-tensor} and \emph{tensor-tensor}, have been omitted.



\begin{definition}
\label{def:equivalence}
%
\emph{Equivalence} of proofs in (cut-free, unit-only) multiplicative linear logic $(\perm)$ is the congruence generated by the permutations given in Figure~\ref{fig:permutations}.
%
\emph{\MLL\ proof equivalence} is the problem of deciding whether two given proofs are equivalent.
%
\end{definition}



The motivation to consider proofs up to equivalence is three-fold.
%
Firstly, there is the strong intuition that the order of permutable inferences does not contribute to the essential content of the proof.
%
Secondly, a technical motivation is that cut-elimination in \MLL\ incorporates permutation steps, and composition via cut-elimination is only associative up to permutations.
%
Thirdly, equivalent proofs are identified in natural models of multiplicative linear logic such as coherence spaces, and in the categorical semantics of \MLL, $\star$-autonomous categories.



In one of several possible definitions, a \emph{$\star$-autonomous category} \citep{Barr-1979} is a symmetric monoidal category $(\mathcal C,\tn,1)$ with:
%
\begin{itemize}
	
	\item
	a \emph{duality}, a contravariant functor $\dual-$ such that $A\cong\dual*A$, and

	\item
	\emph{closure}, an adjunction $-\tn B \dashv \dual{(B\tn\dual-)}$ for any object $B$,

\end{itemize}
%
satisfying natural coherence conditions.
%
%The category $\textsc{mll}(\emptyset)$ of unit-only \MLL-formulae and equivalence classes of proofs is a $\star$-autonomous category.
%
The category with as objects unit-only \MLL-formulae and as morphisms $A\to B$ the equivalence classes of proofs of $\dual A\parr B$, denoted $\textsc{mll}(\emptyset)$, is a $\star$-autonomous category.
%
The present formulation of formulae induces two forms of \emph{strictness}, instances where isomorphisms of the definition are identities: DeMorgan duality means $A=\dual*A$, while
%, one-sided sequents mean the closure adjunction is an equivalence of categories, 
associativity is an identity by decree.
%
Modulo strictness, $\textsc{mll}(\emptyset)$ is the \emph{free} $\star$-autonomous category over the empty category $\emptyset$.
%
This means that \emph{any} $\star$-autonomous category is a model of the logic, and that \MLL\ proof equivalence is the \emph{word problem} for $\star$-autonomous categories, the problem of deciding when two representations of morphisms denote the same morphism.




%The \emph{free $\star$-autonomous completion} $\star$-\textsc a$(\mathcal C)$ of a category $\mathcal C$ is characterised by a functor $i\colon\mathcal C\to\star$-\textsc a$(\mathcal C)$ such that any functor $\mathcal C\to \mathcal D$ into a $\star$-autonomous cateogry $\D$ factors uniquely (up to canonical isomorphism) as the composition of $i$ with a functor $\star$-\textsc a$(\mathcal C)\to\D$ preserving $\star$-autonomous structure.
%%
%\[
%\begin{tikzpicture}[auto]
%	\node (C) at (0,0) {$\mathcal C$};
%	\node (*AC) at (2,0) {$\star$-\textsc a$(\mathcal C)$};
%	\node (D) at (1,1) {$\mathcal D$};
%	\draw [->] (C) -- node {$i$} (*AC);
%	\draw [->] (C) -- (D);
%	\draw [->,dashed] (*AC) -- node {$!$} (D);
%\end{tikzpicture}
%\]
%%



\subsection{Proof nets}

A partial solution to the \MLL\ proof equivalence problem is provided by proof nets.


\begin{definition}
\label{def:proof nets}
%
For a sequent $\Gamma$,
\begin{itemize}

	\item
	a \emph{linking} $\links$ is a function from the names of $\bot$-subformulae to the names of $1$-subformulae,

	\item
	a \emph{switching graph} for $\links$ is an undirected graph over the names of $\Gamma$, with for every subformula $\named aA\named c\tn\named bB$ the edges $a-c$ and $b-c$, for every subformula $\named aA\named c\parr\named bB$ either the edge $a-c$ or the edge $b-c$, and for every subformula $\named a\bot$ the edge $a-\links(a)$,

 	\item
	a \emph{proof net} $\links$ or $(\Gamma,\links)$ is a linking $\links$ such that every switching graph is acyclic and connected.

\end{itemize}
\end{definition}


\noindent
An edge $a-\links(a)$ in a proof net or switching graph is a \emph{link} or \emph{jump}.



\begin{definition}
\label{def:proof net equivalence}
%
A \emph{permutation} between proof nets is the redirection of exactly one link.
%
\emph{Equivalence} $(\perm)$ of proof nets over a sequent $\Gamma$ is the congruence generated by permutations.
%
\end{definition}


\noindent
There is no canonical interpretation of a proof as a proof net, since the introduction rule for $\bot$ in proofs joins a $\bot$-formula to a sequent, rather than a formula.



\begin{definition}
\label{def:proofs to nets}
%
The relation $(\toNet)$ interprets a proof $\Pi$ for a sequent $\Gamma$ by a linking $\links$ as follows:
% 
$\Pi\toNet\links$ if for each $\named a\bot$ in $\Gamma$, if $\Delta$ is the context of the inference introducing $\named a\bot$, as illustrated below, then $\links(a)$ is the name of some $1$ in $\Delta$.
\[
	\infer[\MLLlabel b]{\Delta,\named a\bot}{\Delta}
\]
%
\end{definition}



\begin{proposition}[\citeauthor{DR89}, \citeyear{DR89}]
\label{prop:correctness and sequentialisation}
%
For a proof $\Pi$ with conclusion $\Gamma$, if $\Pi\toNet\links$ then $\links$ is a proof net for $\Gamma$.
%
For a net $\links$ for $\Gamma$, there is a proof $\Pi$ of $\Gamma$ such that $\Pi\toNet\links$ (\emph{sequentialisation}).
%
\end{proposition}


\noindent
Proof nets are canonical representations of proofs in the absence of units: they factor out the permutations among tensor- and par-inferences, which are the last three permutations in Figure~\ref{fig:permutations}.
%
Equivalence of proof nets is generated by the remaining equations, the permutations on $\bot$-introduction.



\begin{proposition}[\citeauthor{HughesMLLProofNets}, \citeyear{HughesMLLProofNets}]
\label{prop:proof nets work}
%
For proofs $\Pi$, $\Pi'$ and proof nets $\links$, $\links'$ such that $\Pi\toNet\links$ and $\Pi'\toNet\links'$, $\Pi\perm\Pi'$ if and only if $\links\perm\links'$.
%
\end{proposition}


\noindent
\MLL\ proof equivalence is the problem of deciding equivalence of proof nets.


\subsection{Notation}


We will use a concise diagrammatic notation for sequents and proof nets.
%
The units $1$ and $\bot$ are represented by a circle $\circ$ and a disc $\bullet$ respectively.
%
A tensor is represented by a line connecting both subformulae, and a par by juxtaposition: if $A$ and $B$ are represented by \raisebox{-0.3\height}{\includegraphics[scale=0.75]{hex-A.pdf}}
and \raisebox{-0.3\height}{\includegraphics[scale=0.75]{hex-B.pdf}},
%
then $A\parr B$ is \raisebox{-0.3\height}{\includegraphics[scale=0.75]{hex-AparB.pdf}}
and $A\tn B$ is \raisebox{-0.3\height}{\includegraphics[scale=0.75]{hex-AtnB.pdf}}.
%
A tensor of multiple elements is denoted by stringing them together in a line, so $A\tn B\tn C$ is
\raisebox{-0.3\height}{\includegraphics[scale=0.75]{hex-AtnBtnC.pdf}}.
%
Boxes play the role of parentheses around par-formulae, so $(A\parr B)\tn C$ is drawn as \begin{center}{\includegraphics[scale=1]{hex-AparBtnC.pdf}}\end{center}

\noindent For example, this sequent
\[ \vdash \bot\tn\bot, \bot\tn\bot, \bot\tn\bot, \bot\tn\bot, 1, \{[(1\parr 1\parr 1)\tn(1\parr 1\parr 1)]\parr 1\parr (\bot\tn\bot\tn\bot)\parr 1\}\tn\bot \]
could be drawn like this:
\begin{center}\includegraphics[scale=0.75]{example-sequent.pdf}\end{center}

\noindent We represent a proof net by drawing an arrow from each $\bullet$ to some $\circ$. For example, one proof net on the above sequent is
\begin{center}\includegraphics[scale=0.75]{example-sequent-proofnet.pdf}\end{center}








%

\newcommand\1{1}
\newcommand\prm[1][\links]{\mathit p_{#1}}


%NOTE: need 'octopus' terminology here

%NOTE: will we need to count arms often enough to make it worthwile to write $|\Gamma|_\bot$ for the number of $\bot$-occurrences in a sequent $\Gamma$?


\section{Equivalence in the absence of $~\protect\parr$}





Let a \emph{1-alternation} sequent be one over formulae of the form $1$ or $\bot\tn\ldots\tn\bot$, where the number of $\bot$-subformulae is at least 2.
%
Such a sequent is inhabited exactly when the number of formulae in the sequent is one greater than the total number of $\bot$-subformulae it contains.
%
An inhabited 1-alternation sequent with only one tensor-formula, i.e.\ a sequent of the form $\1,\ldots,\1,\bot\tn\ldots\tn\bot$ with $n$ $\bot$-subformulae and $n$ $1$-subformulae, will admit $n!$ different proof nets, each with $n$ links.
%
Since no link can re-attach, its equivalence classes are singletons.





\begin{proposition}
\label{lem:level0 max binary}
%
For a 1-alternation sequent with at least two tensor-formulae there are at most two equivalence classes of proof nets.
%
\end{proposition}



\begin{proof}
%
It will be shown by induction on the number of $\bot$-formulae in $\Gamma$ that every proof net for $\Gamma$ belongs to one of two equivalence classes.
%
For the base case, the smallest inhabited sequent with two tensor-formulae is the following.
\[
	\1,\1,\1,\bot\tn\bot,\bot\tn,\bot
\]
It has two equivalence classes of 12 proof nets each.



For the inductive step, let $\Gamma$ be the following sequent.
\[
	\Delta,A\tn\named a\bot,\named x\1
\]
There are two cases: 1) where $A$ is a tensor-formula, and 2) where $A$ is $\bot$ and where, for the induction hypothesis to apply, $\Delta$ contains at least two tensor-formulae.
%
For both cases, it will be shown that any net $\links$ for $\Gamma$ is equivalent to a net $\links'$ where $\named a\bot$ connects to $\named x1$, and is the only link to do so.
%
This reduces equivalence on $\Gamma$ to equivalence on $\Delta,A$ in case 1, and on $\Delta$ in case 2, so that the induction hypothesis applies.


Let $\named a\bot$ connect to $\named y\1$ in $\Delta$, and let $\named c\bot$ be a subformula of $A$, which means that for case 2, $A=\named c\bot$.
%
Then $\links'$ is obtained by adjusting $\links$ as follows.
\begin{itemize}

	\item
Let $c-z$, i.e.\ $\named z\1$ is the target of the jump from $\named c\bot$.	
%
Ensure that $\named z\1$ is the only target shared between jumps in $A$ and in $\Delta$, by moving any other such jump from $\Delta$ to $\named z\1$.

	\item
If there are multiple links connecting to $\named x\1$, select one $b-x$ for some $\named b\bot$ in a tensor-formula $B$.
%
Re-attach the others to the target of another jump out of $B$, of which there must be at least one.

	\item
Since $A$ is only connected via $\named z\1$, there is a jump $d-z$ connecting $B$ to $A$ (though $\named d\bot$ is not necessarily a subformula of $B$).
%
Re-attach $\named d\bot$ to $\named y\1$, then $\named a\bot$ to $\named x\1$, and $\named b\bot$ to $\named z\1$.

\end{itemize}

\end{proof}



Consider the following naming scheme for the units in a 1-alternation sequent $\Gamma$ with tensor-formulae $A_1,\dotsc,A_n$.
%
\begin{itemize}

	\item
The first $\bot$ in each $A_i$ is named $b(i)$, while the remaining $\bot$-subformulae in $\Gamma$ are named $b(n+1),\dotsc,b(m)$.

	\item
The first $\1$ in $\Gamma$ is named by the set $A=\{1,\dotsc,n\}$, the remaining ones are named $n+1,\dotsc,m$.

\end{itemize}
%
The naming scheme suggests a linking for $\Gamma$, where the first $\bot$-subformula of each tensor-formula connects to $\named A\1$, while other $\bot$-subformulae connect uniquely to the remaining $\1$-subformulae; i.e.\ $b(i)-A$ when $i\leq n$ and $b(i)-i$ otherwise.
%
A net for $\Gamma$ can be interpreted as a permutation (an automorphism on $\{1,\dotsc,m\}$) as follows.
%
\begin{definition}
To a proof net $\links$ for a 1-alternation sequent $\Gamma$ named as above, associate the \emph{permutation} $\prm:\{1,\dotsc,m\}\to\{1,\dotsc,m\}$ given by:
\[
	\prm(i) = 
	\begin{cases}
		j 				& \text{ if $b(i)$ may connect to $A$ and $\named{b(i)}\bot$ is a subformula of $t(j)$}
	\\	\links(b(i))	& \text{ otherwise.}
	\end{cases}
\]
The \emph{parity} of $\links$ is the parity of its permutation.
\end{definition}



\begin{proposition}
\label{lem:level0 min binary}
Re-attaching a jump in a net $\links$ preserves its parity. 
\end{proposition}


\begin{proof}
TO DO
\end{proof}
















%
\section{Par}



\begin{definition}
A \emph{sub-sequent} $\Delta\leq\Gamma$ of a sequent $\Gamma$ is a sequent consisting of disjoint subformulae of $\Gamma$, preserving names.
\end{definition}


\begin{definition}
A \emph{subnet} $(\Gamma',\links') \leq (\Gamma,\links)$ of a proof net is a net such that $\Gamma'\leq\Gamma$ and $\links'$ is $(\links|_{\Gamma'})$, the restriction of $\links$ to $\Gamma'$.
\end{definition}


The root vertices of $\Gamma'$ are the \emph{ports} of the sub-sequent $\Gamma'$ and of the subnet $(\Gamma',\links')$.
%
A \emph{scope} of a par $\named v\parr$ is a subnet that has $v$ as a port.
%
In a proof net, the \emph{kingdom} and the \emph{empire} of a par are respectively its smallest and largest scope.



The scopes of a par correspond to the possible subproofs of its introduction rule in a sequentialisation of the proof net that it occurs in.
%
In the graph of a proof net, the scope of a par $\named v\parr$ may be \emph{contracted} to a single vertex $v$ by removing all vertices except $v$ and re-attaching all arcs connecting to removed vertices to $v$.
%

[[ ADD ILLUSTRATED EXAMPLE ]]

%
The contraction of scopes may replace the switching condition as a correctness criterion.
%
The following is a variant of the local retraction algorithm by Danos \cite{Danos-1990}.


\begin{proposition}
\label{prop:scoping correctness}
A linking $\links$ for a sequent $\Gamma$ is a proof net if and only if each $\named v\parr$ is a port of a sub-sequent $s(v)\leq\Gamma$ such that:

\begin{enumerate}
	\item
sub-sequents are strictly nested: if $\named w\parr$ occurs in $s(v)$ then $s(w)<s(v)$; and
	\item
for each $\named v\parr$, the graph $\sigma(v)=(s(v),\links|_{s(v)})$ becomes a tree when all immediate sub-sequents $s(w)$ are contracted.
\end{enumerate}

\end{proposition}


\begin{proof}
For the `if' direction, it follows by induction on the nesting of sub-sequents that each graph $\sigma(v)$ satisfies the switching condition.
%
For the `only if' direction, given a sequentialisation of $(\Gamma,\links)$, a sub-sequent $s(v)\leq\Gamma$ for each $\named v\parr$ is found by taking the conclusion $\Delta, A\named v\parr B$ of its introduction rule, below.
\[
	\infer{\Delta,A\named v\parr B}{\Delta,A,B}
\]
\end{proof}





[[ IDEA: the following could help simplify octopus-arithmetic ]]


\begin{definition}
The \emph{balance} of a sequent is the number of $\bot$s minus the number of $\parr$s and comma's.
%
A sequent is \emph{balanced} if its balance is zero.
\end{definition}

An unbalanced sequent is uninhabited: a positive balance guarantees a cycle in any switching graph, for any linking, while a negative balance similarly guarantees disconnectedness.

An early conjecture of Girard, which turned out to be false, was that a sequent is inhabited if and only if it is balanced. [[FIND CITATION (probably TCS87)]]











%\section{Proof nets and constraint graphs}


\newcommand\itn[1]{\llbracket#1\rrbracket}
\newcommand\coitn[1]{\llparenthesis#1\rrparenthesis}

\begin{definition}
A \emph{non-deterministic constraint graph} (\textsc{ncg}) $G=(V,E,c,v,w)$ consists of a set $V$ of vertices with \emph{minimum inflow constraint} $c\colon V\to\mathbb N$, and a set $E$ of at most one undirected edge $e$ per vertex-pair $v(e)=\{v_1,v_2\}$, with \emph{weight} $w\colon E\to N$.

A \emph{configuration} of an \textsc{ncg} is a partial function $\gamma\colon E\rightharpoonup V$ such that
\begin{itemize}
	\item
for every edge $e$, if $\gamma(e)$ is defined then $\gamma(e)\in v(e)$, and
	\item
for every vertex $v$, the sum of its inflow weights is at least its inflow constraint, $\sum\{w(e)\mid \gamma(e)=v\}\geq c(v)$.
\end{itemize}

A \emph{reconfiguration step} $\gamma\sim\delta$ connects two configurations for an \textsc{ncg} $G$ that differ in the assignment $\gamma(e)$ of exactly one edge.

\end{definition}



[[ NOTE: decide on notation for steps and paths in permutation relations ]]

For a graph $G=(V,E,c,w)$ let $|V|$ and $|E|$ denote the number of vertices and edges, respectively, and let $|c|$ and $|w|$ denote the sum of all inflow constraints, $\sum_{v\in V}c(v)$, and the sum of all edge weights, $\sum_{e\in E}w(e)$.


Let $A^n$ denote the sequent of $n$ copies of a formula $A$, and for a sequent $\Gamma=A_1,\dotsc,A_n$ let $\bigotimes\Gamma=A_1\tn\dotso\tn A_n$ and $\bigparr\Gamma=A_1\parr\dotso\parr A_n$.



\begin{definition}
The \emph{interpretation} $\itn G$ of an \textsc{ncg} $G=(V,E,c,w)$ is a sequent constructed as follows.
%
Let $V=\{v_1,\dotsc,v_n\}$ and $E=\{e_1,\dotsc,e_k\}$ where $|V|=n$ and $|E|=k$.

The interpretation of a vertex $v_i$ is the formula
\[
	\itn{v_i} = \bigparr\big(C^{k\times c(v_i)} \big)\parr 1
\]
where each \emph{constraint element} $C$ is the formula
\[
	C = \bigparr\big(1^{3i+2}\big) \tn \bigparr\big(1^{3(n-i)+3}\big)
\]

The interpretation of an edge $e$ connecting vertices $v_i$ and $v_j$ with $i<j$ is the formula
\[
	\itn{e} = \bigotimes\big(W^{k\times w(e)}\big)\tn\bot
\]
where each \emph{weight element} $W$ is the formula
\[
	W = \bigotimes\big(\bot^{3i+2}\big)\parr\bigotimes\big(\bot^{3(j-i)+1}\big)\parr\bigotimes\big(\bot^{3(n-j)+3}\big)
\]

The interpretation of the graph $G$ is the sequent
\[
	%\itn G = \bigotimes_{1\leq i\leq n}\itn{v_i}, \itn{e_1},\dotsc,\itn{e_k}, \bot^m
	\itn G = \itn{v_1}\tn\dotso\tn\itn{v_n}, \itn{e_1},\dotsc,\itn{e_k}, \bot^m
\]
where $m=k\times(|w|-|c|)\times(3n+4)$.

\end{definition}



\begin{lemma}
In a proof net for $\itn G$, an edge-gadget $\itn e$ belongs to the empire of at most one vertex-gadget $\itn v$.
\end{lemma}

\begin{proof}
Since vertex-gadgets are joined by a tensor, the lemma is immediate from \cite[Proposition 1]{Bellin-vandeWiele-1995}.
\end{proof}


In an \textsc{ncg} $G$, an edge $e$ will be called \emph{appropriate} for a vertex $v$ if $v\in v(e)$, and \emph{inappropriate} otherwise.
%
This notion is extended to edge-gadgets $\itn e$ and vertex-gadgets $\itn v$ in $\int G$.


\begin{lemma}
\label{lem:appropriate edge weights}
In a proof net for $\itn G$, for each vertex $v$, the weights of the appropriate edge-gadgets in the empire of $\itn v$ are equal to or greater than the constraint of $v$.
\end{lemma}



\begin{proof}[Proof sketch]
There are not enough edges to fill a unit of weight inappropriately in $\itn v$.
\end{proof}


Then using the above, a proof net for $\itn G$ may be interpreted as a configuration for $G$.



\begin{definition}
	For a proof net $\links$ for the interpretation of a graph $\itn G$, let $\coitn\links$ be the configuration for $G$ where $\coitn\links(e)$ is $v$ if 1) $e$ is appropriate for $v$ and 2) $\itn e$ belongs to the empire of $\itn v$, and undefined otherwise.
\end{definition}





\begin{lemma}
If $\links\sim\links'$ are proof nets for $\itn G$ then $\coitn\links\sim\coitn{\links'}$.
\end{lemma}

\begin{proof}
A single permutation $\links\sim\links'$ on proof nets may move a number of edge-gadgets $\itn{e_1},\dotso,\itn{e_n}$ in three ways: 1) out of the empire of a vertex-gadget $\int v$, 2) into the empire of a vertex-gadget $\int w$, or 3) both.
%
Then in both $\coitn\links$ and $\coitn{\links'}$ the edges $e_1$ through $e_n$ are mobile, since by Lemma~\ref{lem:appropriate edge weights} the empires of the vertex-gadgets $\itn v$ (in cases 1 and 3) and $\itn w$ (in cases 2 and 3) contain appropriate edge-gadgets other than $\itn{e_1}$ through $\itn{e_n}$ of sufficient combined weight.
%
It follows that in the graph $G$ the edges $e_1$ through $e_n$ can be moved away from $v$ and/or onto $w$ one at a time.

\end{proof}



%
%
%
%\begin{definition}
%The interpretation $\itn\gamma$ of a configuration $\gamma$ for a graph $G$ is the set of proof nets for $\itn G$ satisfying the following constraint: if $\gamma(e)=v$ then $\itn e$ belongs to the empire of $\itn v$
%\end{definition}









%\include{s_conclusion}



\bibliographystyle{plainnat}
\bibliography{refs}

\end{document}







