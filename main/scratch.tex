\documentclass[conference]{IEEEtran}


\usepackage{amsmath,amsthm}
\usepackage{mllequiv,willemtools}

\usepackage{pgffor}

\graphicspath{{../notes/}}

%\makeTheoremDefs


\begin{document}

\[
\begin{tikzpicture}[x=-5mm,y=5mm,octo]
	\node[big_] (A) at (-1,2) {};
	\node[big1] (!) at (-1,0) {};
	\node at (.5,1) {$\Rightarrow$};
	\draw (2,2) node[bullet] (a) {} -- (3,2) node[bullet] (b) {} -- (4,2) (5,2) -- (6,2) node[bullet] (c) {}; 
	\node[circ] (1) at (2,0) {}; \node[circ] (2) at (3,0) {}; \node[circ] (3) at (6,0) {};
	\draw[dotted] (4,2)--(5,2) (4,0)--(5.1,0);
	\begin{scope}[thepink,->]
			\draw[big>] (A) to (!.center);
			\draw[bend right=10] (a) to (1.center);
			\draw[bend right=10] (b) to (2.center);
			\draw[bend right=10] (c) to (3.center);
	\end{scope}
\end{tikzpicture}
\] 	

\[
\vc{
  \begin{tikzpicture}[x=5mm,y=-5mm,octo]
	\draw (0,0) node[bullet] (a) {}--
		  (1,0) node[bullet] (b) {}--
		  (2,0) node[bullet] (c) {};
	\draw (4,0) node[bullet] (d) {}--
		  (5,0) node[bullet] (e) {}--
		  (6,0) node[bullet] (f) {}--
		  (7,0) node[bullet] (g) {};
	\foreach \i in {1,...,6} {\node[circ] (\i) at (\i,3) {};}
	\begin{scope}[thepink,->]
			\draw[bend right=10] (a) to (1);
			\draw[bend right=10] (b) to (2);
			\draw[bend right=10] (c) to (3);
			\draw[bend left=10]  (d) to (3);
			\draw[bend left=10]  (e) to (4);
			\draw[bend left=10]  (f) to (5);
			\draw[bend left=10]  (g) to (6);
	\end{scope}
  \end{tikzpicture}
}
\quad\Rightarrow\quad
\vc{
  \begin{tikzpicture}[x=5mm,y=-5mm,octo]
	\node[big_] (A) at (-1,0) {$2'$};
	\node[big_] (B) at ( 1,0) {$3'$};
	\node[big1] (!) at ( 0,3) {$5'$};
	\begin{scope}[thepink,->,big>]
		\draw[bend right=15] (A) to (!);
		\draw[bend left=15]  (B) to (!);
	\end{scope}
  \end{tikzpicture}
}
\]


%
%\[
%\begin{tikzpicture}[x=5mm,y=-5mm,octo]
%	\draw (-1.75,0) node[bullet] (x) {} -- (-.75,0) node[bullet] (y) {}
%		  (  .75,0) node[bullet] (a) {} -- (1.75,0) node[bullet] (b) {}
%		  ( 3.25,0) node[bullet] (c) {} -- (4.25,0) node[bullet] (d) {};
%	\node[circ] (0) at (5.75,0) {};
%	\node[circ] (1) at (-1,3) {};
%	\node[circ] (2) at (1,3) {}; \node[circ] (3) at (2,3) {}; \node[circ] (4) at (3,3) {};
%	\draw[rounded corners] (0,2) rectangle (4,4);
%	\draw (4,3) -- (5,3) node[bullet] (e) {};
%	\begin{scope}[thepink,->]
%			\draw[bend right=10] (x) to (1.center);
%			\draw[bend right=20] (y) to (2.center);
%			\draw[bend right=10] (a) to (2.center);
%			\draw[bend right=10] (b) to (3.center);
%			\draw[bend left=10]  (c) to (3.center);
%			\draw[bend left=10]  (d) to (4.center);
%			\draw[bend right=10] (e) to (0.center);
%	\end{scope}
%\end{tikzpicture}
%\]

\end{document}


\begin{frame}
	\octorollA1
\end{frame}

\begin{frame}
	\octorollB1
\end{frame}

\begin{frame}
	\octorollB2
\end{frame}

\begin{frame}
	\octorollB3
\end{frame}

\begin{frame}
	\octorollC
\end{frame}

\begin{frame}
	\octorollD1
\end{frame}
%
%\begin{frame}
%	\octorollE1
%\end{frame}
%
%\begin{frame}
%	\octorollE2
%\end{frame}
%
%\begin{frame}
%	\octorollE3
%\end{frame}
%
%\begin{frame}
%	\octorollE4
%\end{frame}
%
%\begin{frame}
%	\octorollE5
%\end{frame}
%
%\begin{frame}
%	\octorollE6
%\end{frame}
%
%\begin{frame}
%	\octorollE7
%\end{frame}

\begin{frame}
	\octorollD2
\end{frame}

\begin{frame}
	\octorollA2
\end{frame}

\end{document}




%\newcommand\octorollC{
%  \begin{tikzpicture}[x=5mm,y=-5mm,octo]
%	\node[circ] (a1a) at (0,1.5) {}; \node[big1] (a1) at (-2,0) {}; \node[big1] (a2) at (0,0) {}; \node[big1] (a3) at (2,0) {};
%	\draw (-9  ,4) node[bullet] (d1a) {} -- (-8  ,4) node[big_] (d1) {};
%	\draw (-6.5,4) node[bullet] (d2a) {} -- (-5.5,4) node[big_] (d2) {};
%	\draw (-4  ,4) node[bullet] (d3a) {} -- (-3  ,4) node[big_] (d3) {};
%	\draw[rounded corners] (-10,3) rectangle (-2,5);
%	\draw (-2,4)--(-1,4) node[bullet] (d) {};
%	\draw ( 3  ,4) node[bullet] (e1a) {} -- ( 4  ,4) node[big_] (e1) {};
%	\draw ( 5.5,4) node[bullet] (e2a) {} -- ( 6.5,4) node[big_] (e2) {};
%	\draw ( 8  ,4) node[bullet] (e3a) {} -- ( 9  ,4) node[big_] (e3) {};
%	\draw[rounded corners] (10,3) rectangle (2,5);
%	\draw (2,4)--(1,4) node[bullet] (e) {};
%	\node[circ] (v1a) at (-4,8); \node[big1] (v1b) at (-3,8) {}; \node[big1] (v1c) at (-1.5,8) {};
%	\node[big1] (v2) at  ( 2.5,8) {}; \node[big1] (v2) at ( 2.5,8) {};
%	\draw[rounded corners] (-4.5,7) rectangle (-0.5,9);
%	\draw[rounded corners] ( 3.5,7) rectangle ( 0.5,9);
%	\draw (-0.5,8)--(0.5,8);
%	\node[circ] (v) at (0,6) {};
%	%
%	\begin{scope}[thepink,->]
%		\draw[bend left=15] (e) to (v);\draw[bend right=15] (d) to (v);
%		\draw[big>,bend right] (d1) to (v1); \draw[big>,bend right=20] (d2) to (v1); \draw[big>, bend right=20] (d3) to (v2);
%		\draw[bend right=15] (e1a) to (a1a); \draw[bend right=15] (e2a) to (a1a); \draw[bend right=15] (e3a) to (a1a); 
%		\draw[big>,bend right=20] (e1) to (a1);  \draw[big>,bend right=20] (e2) to (a2);  \draw[big>, bend right=20] (e3) to (a3);
%	\end{scope}
%  \end{tikzpicture}
%}

\[
	\big(\bigparr_{k\times c(v_i)}\big(\bigparr_{3i+2}1 \tn \bigparr_{3(n-i)+3}1\big) \big)\parr 1
\]
\[
	\big(\bigotimes_{k\times w(e)}\big(\bigotimes_{3i+2}\bot\parr\bigotimes_{3(j-i)+1}\bot\parr\bigotimes_{3(n-j)+3}\bot\big)\big)\tn\bot
\]


\[
	C = \bigparr(1^{3i+2}) \tn \bigparr(1^{3(n-i)+3})
\]

\[
	\big(\bigparr C^{k\times c(v_i)} \big)\parr 1
\]

\[
	W = \bigotimes(\bot^{3i+2})\parr\bigotimes(\bot^{3(j-i)+1})\parr\bigotimes(\bot^{3(n-j)+3}
\]

\[
	\big(\bigotimes W^{k\times w(e)}\big)\tn\bot
\]
