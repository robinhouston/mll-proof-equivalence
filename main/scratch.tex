\documentclass[conference]{IEEEtran}


\usepackage{amsmath,amssymb,amsthm}
\usepackage{mllequiv,willemtools}

\usepackage{pgffor}

\graphicspath{{../notes/}}

\makeTheoremDefs




\begin{document}





%
%\begin{proposition}
%\label{prop:level0 may-connect path}
%%
%For a proof net for a 1-alternation sequent containing a link $\edge{\named a\bot}{\named b\1}$ and a formula $\named c\1$, the following are equivalent.
%%
%\begin{itemize}
%	\item
%The edge $\edge ab$ can be permuted to $\edge ac$.
%	\item
%There is a path from $b$ to $c$ not passing through $a$.
%	\item
%The path from $a$ to $c$ starts with the jump $a-b$.
%\end{itemize} 
%\end{proposition}




If a link $\edge ab$ may be reconnected as $\edge ac$ it is said that $a$ \emph{may connect to} $c$. 
%
By the above proposition, it is immediate that if $a$ and $b$ may both connect to $c$, then after actually reconnecting $\edge ac$, still $b$ may connect to $c$.



Consider the following naming scheme for the units in a 1-alternation sequent $\Gamma$ with tensor-formulae $A_1,\dotsc,A_n$.
%
\begin{itemize}

	\item
One $\1$ in $\Gamma$ is named $*$, and the remaining ones with the numbers $n+1,\dotsc,m$.

	\item
A $\bot$-formula in $A_i$ is named by a pair $(i,k)$, where $k=i$ for the first $\bot$-formula in each $A_i$, and for the remaining $\bot$-formulae in all $A_i$, each $k$ is a distinct number in $n+1,\dotsc,m$.

\end{itemize}
%
The naming scheme suggests a linking for $\Gamma$, defined by $\links(i,i)=\star$ and $\links(i,k)=k$ otherwise; i.e\ the first $\bot$ in each tensor-formula connects to $\named *\1$, while other $\bot$-subformulae connect uniquely to the remaining $\1$-subformulae.



A net for $\Gamma$ is interpreted as a combinatorial permutation (an automorphism on $\{1,\dotsc,m\}$) as follows.
%
\begin{definition}
\label{def:combinatorial permutation}
To a proof net $\links$ for a 1-alternation sequent $\Gamma$ named as above, associate the \emph{permutation} $\prm:\{1,\dotsc,m\}\to\{1,\dotsc,m\}$ given by:
\[
	\prm(k) = 
	\begin{cases}
		i				& \text{ if $(i,k)$ may connect to $*$; and}
	\\	\links(i,k)		& \text{ otherwise.}
	\end{cases}
\]
The \emph{parity} of $\links$ is the parity of its permutation.
\end{definition}


To see that $\prm$ is injective, consider the following.
\begin{itemize}
	\item The domains of $i$ and $\links(i,k)$, respectively $1,\dotsc,n$ and $n+1,\dotsc,m$, are disjoint.
	\item Exactly one $\bot$-formula in each $A_i$ may connect to $*$ because of connectedness and acyclicity, since if a $\bot$-formula may connect to $*$ it has a path to $*$ (Lemma~\ref{lem:rewiring within empire}).
	\item If two $\bot$-formulae have the same target, which means they are in different tensor-formulae, at least one may connect to $*$ via the other tensor-formula, which must have a path to $*$ by the above.
\end{itemize}



\begin{proposition}
\label{prop:level0 min binary}
A permutation on a net $\links$ preserves its parity. 
\end{proposition}


\begin{IEEEproof}
Let $\links$ be a net for $\Gamma$, with $\Gamma$ named as above, and let the link $\edge{(i,k)}x$ in $\links$ re-attach as $\edge{(i,k)}y$, forming $\links'$.
%
There are two cases, depending on whether $(i,k)$ may connect to $*$.
%
If so, using Lemma~\ref{lem:rewiring within empire}, the re-wiring preserves which $\bot$-formulae may connect to $*$, since for any path to $*$ via $\edge{(i,k)}x$ in $\links$ there is a path to $*$ via $\edge{(i,k)}y$.
%
Then the permutation of $\links'$ is that of $\links$.


If $(i,k)$ may not connect to $*$, let the path from $x$ to $y$ run via the following $\bot$- and $\1$-vertices.
\[
	x=x_1, (i_1,j_1), (i_1,k_1), x_2, (i_2,j_2), \dotsc, (i_n,k_n), x_{n+1}=y 	
\]
Note that the $\bot$-formulae $(i_a,j_a)$ may connect to $*$.
%
On the relevant domain, this gives the following permutation for $\links$.
\[
\left(\begin{array}{ccccccc}
	j_1 & \dotso & j_n &  k  & k_1 & \dotso & k_n \\
	i_1 & \dotso & i_n & x_1 & x_2 & \dotso & x_{n+1}
\end{array}\right)
\]
In $\links'$, since $\links'(i,k)=y$, the $\bot$-formulae that may connect to $*$ are the $(i_a,k_a)$.
%
The permutation $\prm[\links']$ is the following.
\[
\left(\begin{array}{ccccccc}
	j_1 & \dotso & j_n &    k    & k_1 & \dotso & k_n \\
	x_1 & \dotso & x_n & x_{n+1} & i_1 & \dotso & i_n
\end{array}\right)
\]
The parity of both permutations is the same if and only if the relative permutation, below, is even.
\[
\left(\begin{array}{ccccccc}
	i_1 & \dotso & i_n & x_1     & x_2 & \dotso & x_{n+1} \\
	x_1 & \dotso & x_n & x_{n+1} & i_1 & \dotso & i_n
\end{array}\right)
\]
This is the case, as it is obtained by the exchange of $x_a$ and $i_a$ for each $a\leq n$, and subsequently the exchange of $x_{n+1}$ and each $i_a$ in turn.
%
\end{IEEEproof}



\begin{proposition}
\label{prop:parity determines equivalence}
Two proof nets for a 1-alternation sequent with at least two tensor-formulae are equivalent if and only if they have the same parity.
\end{proposition}


\begin{theorem}
\MLL\ proof equivalence in the absence of ~$\parr$ is linear-time decidable.
\end{theorem}

\begin{IEEEproof}
For a sequent with 1 tensor-formula, the problem is reduced to syntactic equality.
%
For a sequent with 2 or more tensor-formulae, by Propositions~\ref{prop:parity determines equivalence} the equivalence of two nets is determined by their parity.
%
Following Definition~\ref{def:combinatorial permutation} the parity of a net can be read off in a single traversal of the net.
%
This yields a linear-time algorithm.
%
\end{IEEEproof}



\end{document}


%
%
%
%\begin{lemma}
%For a proof net $(\Gamma,\links)$, if the linking $\links'$ agrees with $\links$ on all values except $\links'(a)=v$, then $\links'$ is a proof net if and only if $v$ is in the empire of $a$ in $(\Gamma,\links)$.
%%
%%If $\links\perm*\links'$ by rewiring the link $\links(a)=v$ to $\links'(a)=w$, then $w$ is in the empire of $a$ in the proof net $\links$.
%\end{lemma}
%
%\begin{proof}
%The empire of $a$ corresponds to the largest subproof $\Sigma$ in any $\Pi\toNet\links$ with as conclusion the introduction rule of $\named a\bot$.
%%
%In one direction, by Definition~\ref{def:proofs to nets} $a$ may link anywhere in a translation of $\Sigma$ to a net; if $\named v\1$ is in $\Sigma$, then $\links'$ is a net.
%%
%In the other direction, if $\links\perm*\links'$, then for any switching of $(\Gamma,\links')$ there is a path from $\links(a)=w$ to 
%
%%Let $\sigma$ be the empire of $a$ in $(\Gamma,\links)$.
%%
%In one direction, if $v$ is in $\sigma$, then since $\sigma$ is a subnet, for any switching of $\sigma$ there is a path from $\links(a)$ to $v$ not passing through $a$.
%%
%Then if $\sigma'\perm*\sigma$ by 
%
% $\sigma$ minus the link $\edge a{\links(a)}$ is a subnet, 
%
%By definition, $\sigma$ is the largest subnet with $a$ as a port.
%%
%Then there is a proof $\Sigma\tonet\sigma$ with as conclusion the introduction rule for $\named a\bot$, which 1) is a subproof of some sequent proof $\Pi\tonet\links$, and 2) is the largest subproof of $\named a\bot$ among all $\Pi\tonet\links$.
%
%
%In one direction, if $\links'$ is a proof net, then $\links\perm*\links'$, and there is a $\Pi'\tonet\links'$ such that $\Pi\perm\Pi'$ by permuting the introduction rule for $\named a\bot$.
%%
%But since permuting this rule in $\Pi$ cannot give it a larger subnet than $\Sigma$ (by 2 above), $\named v\1$ must be in $\Sigma$, and $v$ in $\sigma$.
%
%By Definition~\ref{def:proofs to nets} 

\end{proof}



\end{document}
%
%
%\begin{frame}
%	\octorollA1
%\end{frame}
%
%\begin{frame}
%	\octorollB1
%\end{frame}
%
%\begin{frame}
%	\octorollB2
%\end{frame}
%
%\begin{frame}
%	\octorollB3
%\end{frame}
%
%\begin{frame}
%	\octorollC
%\end{frame}
%
%\begin{frame}
%	\octorollD1
%\end{frame}
%
%\begin{frame}
%	\octorollE1
%\end{frame}
%
%\begin{frame}
%	\octorollE2
%\end{frame}
%
%\begin{frame}
%	\octorollE3
%\end{frame}
%
%\begin{frame}
%	\octorollE4
%\end{frame}
%
%\begin{frame}
%	\octorollE5
%\end{frame}
%
%\begin{frame}
%	\octorollE6
%\end{frame}
%
%\begin{frame}
%	\octorollE7
%\end{frame}

\begin{frame}
	\octorollD2
\end{frame}

\begin{frame}
	\octorollA2
\end{frame}

\end{document}




%\newcommand\octorollC{
%  \begin{tikzpicture}[x=5mm,y=-5mm,octo]
%	\node[circ] (a1a) at (0,1.5) {}; \node[big1] (a1) at (-2,0) {}; \node[big1] (a2) at (0,0) {}; \node[big1] (a3) at (2,0) {};
%	\draw (-9  ,4) node[bullet] (d1a) {} -- (-8  ,4) node[big_] (d1) {};
%	\draw (-6.5,4) node[bullet] (d2a) {} -- (-5.5,4) node[big_] (d2) {};
%	\draw (-4  ,4) node[bullet] (d3a) {} -- (-3  ,4) node[big_] (d3) {};
%	\draw[rounded corners] (-10,3) rectangle (-2,5);
%	\draw (-2,4)--(-1,4) node[bullet] (d) {};
%	\draw ( 3  ,4) node[bullet] (e1a) {} -- ( 4  ,4) node[big_] (e1) {};
%	\draw ( 5.5,4) node[bullet] (e2a) {} -- ( 6.5,4) node[big_] (e2) {};
%	\draw ( 8  ,4) node[bullet] (e3a) {} -- ( 9  ,4) node[big_] (e3) {};
%	\draw[rounded corners] (10,3) rectangle (2,5);
%	\draw (2,4)--(1,4) node[bullet] (e) {};
%	\node[circ] (v1a) at (-4,8); \node[big1] (v1b) at (-3,8) {}; \node[big1] (v1c) at (-1.5,8) {};
%	\node[big1] (v2) at  ( 2.5,8) {}; \node[big1] (v2) at ( 2.5,8) {};
%	\draw[rounded corners] (-4.5,7) rectangle (-0.5,9);
%	\draw[rounded corners] ( 3.5,7) rectangle ( 0.5,9);
%	\draw (-0.5,8)--(0.5,8);
%	\node[circ] (v) at (0,6) {};
%	%
%	\begin{scope}[thepink,->]
%		\draw[bend left=15] (e) to (v);\draw[bend right=15] (d) to (v);
%		\draw[big>,bend right] (d1) to (v1); \draw[big>,bend right=20] (d2) to (v1); \draw[big>, bend right=20] (d3) to (v2);
%		\draw[bend right=15] (e1a) to (a1a); \draw[bend right=15] (e2a) to (a1a); \draw[bend right=15] (e3a) to (a1a); 
%		\draw[big>,bend right=20] (e1) to (a1);  \draw[big>,bend right=20] (e2) to (a2);  \draw[big>, bend right=20] (e3) to (a3);
%	\end{scope}
%  \end{tikzpicture}
%}

\[
	\big(\bigparr_{k\times c(v_i)}\big(\bigparr_{3i+2}1 \tn \bigparr_{3(n-i)+3}1\big) \big)\parr 1
\]
\[
	\big(\bigotimes_{k\times w(e)}\big(\bigotimes_{3i+2}\bot\parr\bigotimes_{3(j-i)+1}\bot\parr\bigotimes_{3(n-j)+3}\bot\big)\big)\tn\bot
\]


\[
	C = \bigparr(1^{3i+2}) \tn \bigparr(1^{3(n-i)+3})
\]

\[
	\big(\bigparr C^{k\times c(v_i)} \big)\parr 1
\]

\[
	W = \bigotimes(\bot^{3i+2})\parr\bigotimes(\bot^{3(j-i)+1})\parr\bigotimes(\bot^{3(n-j)+3}
\]

\[
	\big(\bigotimes W^{k\times w(e)}\big)\tn\bot
\]
