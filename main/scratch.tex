\documentclass[conference]{IEEEtran}
%
%\usepackage[default]{frcursive}
%\usepackage[T1]{fontenc}

\usepackage{amsmath,amssymb,amsthm}
\usepackage{mllequiv,willemtools}


%\DeclareRobustCommand{\frcshape}{\fontfamily{frc}\selectfont}
%\DeclareTextFontCommand{\textfrc}{\frcshape}


\makeTheoremDefs


\begin{document}

\begin{figure}
$
\quad
\vc{\begin{tikzpicture}[x=3mm,y=3mm]
	\draw[ncl2,->-] (0,0) -- (3,0);
	\draw[ncl1,->-] (3,0) -- (5,2);
	\draw[ncl1,->-] (3,0) -- (5,-2);
	\node[nclnode] at (3,0) {};
\end{tikzpicture}}
\,\perm*\hfill
\vc{\begin{tikzpicture}[x=3mm,y=3mm]
	\draw[ncl2,->-] (0,0) -- (3,0);
	\draw[ncl1,-<-] (3,0) -- (5,2);
	\draw[ncl1,->-] (3,0) -- (5,-2);
	\node[nclnode] at (3,0) {};
\end{tikzpicture}}
\,\perm*\hfill
\vc{\begin{tikzpicture}[x=3mm,y=3mm]
	\draw[ncl2,->-] (0,0) -- (3,0);
	\draw[ncl1,-<-] (3,0) -- (5,2);
	\draw[ncl1,-<-] (3,0) -- (5,-2);
	\node[nclnode] at (3,0) {};
\end{tikzpicture}}
\,\perm*\hfill
\vc{\begin{tikzpicture}[x=3mm,y=3mm]
	\draw[ncl2,-<-] (0,0) -- (3,0);
	\draw[ncl1,-<-] (3,0) -- (5,2);
	\draw[ncl1,-<-] (3,0) -- (5,-2);
	\node[nclnode] at (3,0) {};
\end{tikzpicture}}
\quad
$

\smallskip
\caption{A series of reconfiguration steps in a constraint graph}
\label{fig:NCL example}
\end{figure}

\begin{figure}
\[
\scalebox{0.6}{
\begin{tikzpicture}[x=5mm,y=5mm,octo]
	\node[big_] (A) at (0,0) {}; 
	\node[big_] (B) at (2,0) {};
	\node[big_] (C) at (4,0) {};
	\draw[rounded corners] (-1,-1) rectangle (5,1);
	\node[big_] (D) at (0,4) {}; 
	\node[big_] (E) at (2,4) {};
	\node[big_] (F) at (4,4) {};
	\draw[rounded corners] (-1,3) rectangle (5,5);
	\draw (2,1)--(2,1.5) (2,2.5)--(2,3);
	\draw[dotted] (2,1.5)--(2,2.5);
	\draw (2,-1)--(2,-2) node[bullet] (e1) {};
	%%
	\node[circ] (v) at (9.5,-2) {};
	\node[big1] (1) at (8,0) {}; 
	\node[big1] (2) at (11,0) {};
	\draw[rounded corners] (7,-1) rectangle (9,1) (10,-1) rectangle (12,1);
	\draw (9,0)--(10,0);
	\node[big1] (3) at (8,4) {};
	\node[big1] (4) at (11,4) {};
	\draw[rounded corners] (7,3) rectangle (9,5) (10,3) rectangle (12,5);
	\draw (9,4)--(10,4);
	\draw[dotted] (9.5,1.5)--(9.5,2.6);
	\node[big1] (5) at (8,7) {};
	\node[big1] (6) at (11,7) {};
	\draw[rounded corners] (7,6) rectangle (9,8) (10,6) rectangle (12,8);
	\draw (9,7)--(10,7);
	\node[big1] (7) at (8,11) {};
	\node[big1] (8) at (11,11) {};
	\draw[rounded corners] (7,10) rectangle (9,12) (10,10) rectangle (12,12);
	\draw (9,11)--(10,11);
	\draw[dotted] (9.5,8.5)--(9.5,9.6);
	\node[big1] (9) at (9.5,14) {};
	\node[big1] (10) at (9.5,18) {};
	\draw[rounded corners] (6.5,-2.5) rectangle (12.5,12.5);
	%%
	\draw (17,6)--(17,-2) node[bullet] (e2) {};
	\node[big_] (G) at (15,7) {}; 
	\node[big_] (H) at (17,7) {};
	\node[big_] (I) at (19,7) {};
	\draw[rounded corners] (14,6) rectangle (20,8);
	\node[big_] (J) at (15,11) {}; 
	\node[big_] (K) at (17,11) {};
	\node[big_] (L) at (19,11) {};
	\draw[rounded corners] (14,10) rectangle (20,12);
	\draw (17,8)--(17,8.5) (17,9.5)--(17,10);
	\draw[dotted] (17,8.5)--(17,9.5);
	\draw (17,12)--(17,13);
	\node[big_] (M) at (15,14) {}; 
	\node[big_] (N) at (17,14) {};
	\node[big_] (O) at (19,14) {};
	\draw[rounded corners] (14,13) rectangle (20,15);
	\node[big_] (P) at (15,18) {}; 
	\node[big_] (Q) at (17,18) {};
	\node[big_] (R) at (19,18) {};
	\draw[rounded corners] (14,17) rectangle (20,19);
	\draw (17,15)--(17,15.5) (17,16.5)--(17,17);
	\draw[dotted] (17,15.5)--(17,16.5);
	%%
	\begin{scope}[thepink,->]
		\draw[bend left=10] (e1) to (v);
		\draw[bend right=10] (e2) to (v);
	\end{scope}
	\begin{scope}[thepink,->,big>]
		\draw[bend left=30]  (A) to (1);
		\draw[out=30,in=165] (B) to (1);
		\draw[bend left=30]  (C) to (2);
		\draw[bend left=30]  (D) to (3);
		\draw[out=30,in=165] (E) to (3);
		\draw[bend left=30]  (F) to (4);
		\draw[bend right=30] (G) to (5);
		\draw[in=15,out=150] (H) to (6);
		\draw[bend right=30] (I) to (6);
		\draw[bend right=30] (J) to (7);
		\draw[in=15,out=150] (K) to (8);
		\draw[bend right=30] (L) to (8);
		\draw (M) to (9);
		\draw[in=15,out=150] (N) to (9);
		\draw[bend right=30] (O) to (9);
		\draw (P) to (10);
		\draw[in=15,out=150] (Q) to (10);
		\draw[bend right=30] (R) to (10);
	\end{scope}
\end{tikzpicture}}
\]
\caption{Two edge-gadgets connecting to a vertex-gadget}
\label{fig:connection example}
\end{figure}


\end{document}

\tikzstyle{octar}=[thepink,->,octo]
\tikzstyle{pn}=[inner sep=1pt]
\newcommand\nx[2]{#2_{\smash{\makebox[0pt][l]{\ensuremath{\scriptstyle{#1}}}}}}

Given a named sequent $\Gamma$, we can identify edges across all switching graphs for all proof nets for $\Gamma$ by their parent connective: a tensor $\bigtn(A_1,\dotsc,A_n)$ named $v$ has $n$ edges $v(1),\dotsc,v(n)$; a par $\named v\pr$ has one switched edge $v(1)$; and $\named v\bot$ has the jump $v(1)$.
%
For a switching graph, selecting a vertex as the root establishes a bijection between edges and the remaining nodes, which associates each edge with its target.
%
The example below displays a proof net on the left, and on the right the switching graph choosing the edge $\edge ij$ for the par $\named i\pr$, with chosen root $g$.
\[
	\vc{\parityNetA} \qquad \vc{\parityNetB}
\]
The tree above right associates, for example, the edge $c(3)$ with the vertex $c$, and $i(1)$ with $j$.
%
Two proof nets for the same sequent establish two such bijections, composing which gives a permutation on the (non-root) vertices.
%
If the nets are equivalent, the parity of this permutation is even.
%
This follows from what will be established next: a single rewiring connects switching graphs with even parity, and so does switching a single par.
%
The case for rewiring is illustrated; that for switching is similar.
%
For rewiring a jump $\edge va$ to $\edge vb$ (named $v(1)$ in the above scheme) there are two possibilities.
%
If the jump is directed towards the root, illustrated below left, then each node is connected to the root via the same edge, and the permutation is the identity.
%
If the jump is directed downward, the subtree of $a$ will get the new root node $b$, as illustrated below right.
%
\[
	\vc{\parityNetC1} \perm \vc{\parityNetC2}
\quad \vc{\rule{.5pt}{4cm}} \quad
	\vc{\parityNetD1} \perm \vc{\parityNetD2} \kern5pt{=}\kern-5pt \vc{\parityNetD3}
\]
%
The vertices that are assigned a new edge are exactly those on the path from $a$ to $b$ in the switching graph, illustrated below.
%
\[
	\parityNetE
\]
%
Since proof nets were assumed to be strictly alternating---each even $v_i$ is a $\bot$ or $\pr$, each odd $v_j$ a $\1$ or $\tn$---there are an odd number of vertices on this path, $2n+3$.
%
The permutation it induces is a cyclic one, matching each $v_i$ with $v_{i+1}$, $v_{2n}$ with $b$, $b$ with $a$, and $a$ with $v_0$.
%
A cyclic permutation of odd length is an even permutation.


\end{document}



%
%\newcommand\parityNetA{
%  \begin{tikzpicture}[x=6mm,y=-7mm]
%	\node[pn] (a) at (-2,2) {$\nx a\1$};
%	\node[pn] (b) at (-1,2) {$\nx b\1$};
%	\node[pn] (c) at (1,2) {$\nx c\tn$};
%	\node[pn] (d) at (0,1) {$\nx d\bot$};
%	\node[pn] (e) at (1,1) {$\nx e\bot$};
%	\node[pn] (f) at (2,1) {$\nx f\bot$};
%	\node[pn] (g) at (4,2) {$\nx g\tn$};
%	\node[pn] (h) at (3,1) {$\nx h\bot$};
%	\node[pn] (i) at (5,1) {$\nx i\pr$};
%	\node[pn] (j) at (4,0) {$\nx j\1$};
%	\node[pn] (k) at (6,0) {$\nx k\1$};
%	%
%	\draw (c)--(d) (c)--(e) (c)--(f) (g)--(h) (g)--(i);
%	\draw[thick,dash pattern=on 3pt off 2pt] (i)--(j) (i)--(k);
%	%
%	\draw[octar,bend right=45] (d) to (b);
%	\draw[octar,bend left=60]  (e) to (k);
%	\draw[octar,bend left=45]  (f) to (j);
%	\draw[octar,bend right=60] (h) to (a);
%  \end{tikzpicture}
%}
%
%\newcommand\parityNetB{
%  \begin{tikzpicture}[x=6mm,y=-4mm]
%	\node[pn] (a) at (-1,3) {$\nx a\1$};
%	\node[pn] (b) at (0,10) {$\nx b\1$};
%	\node[pn] (c) at (1,7) {$\nx c\tn$};
%	\node[pn] (d) at (0,8) {$\nx d\bot$};
%	\node[pn] (e) at (2,8) {$\nx e\bot$};
%	\node[pn] (f) at (1,5) {$\nx f\bot$};
%	\node[pn] (g) at (0,0) {$\nx g\tn$};
%	\node[pn] (h) at (-1,1) {$\nx h\bot$};
%	\node[pn] (i) at (1,1) {$\nx i\pr$};
%	\node[pn] (j) at (1,3) {$\nx j\1$};
%	\node[pn] (k) at (2,10) {$\nx k\1$};
%	%
%	\draw (c)--(d) (c)--(f) (g)--(h);
%	\draw ($(g)!0.45!(i)$) -- (i);
%	\draw ($(c)!0.4!(e)$) -- (e);
%	\draw[thick,dash pattern=on 2pt off 2pt] (i)--(j);
%	%
%	\draw[octar,bend right=45] (d) to (b);
%	\draw[octar,bend right=45]  (e) to (k);
%	\draw[octar,bend left=45]  (f) to (j);
%	\draw[octar,bend right=45] (h) to (a);
%  \end{tikzpicture}
%}
%
%\newcommand\parityNetC[1]{
%	\begin{tikzpicture}[x=5mm,y=-4mm,octo]
%	  \node[pn] (1) at (-1,2) {$\nx a\1$};
%	  \node[pn] (2) at ( 1,2) {$\nx b\1$};
%	  \node[pn] (a) at ( 0,5) {$\bot$};
%	  %
%	  \draw (-.5,2) -- (2) -- (0,0) -- (1);
%	  \draw (a) -- (1,7) -- (-1,7) -- (a);
%	  %
%	  \SWITCH{#1}
%	    {1}{\draw[thepink,->,bend right=25] (a) to ($(1)+(.5,.5)$);}
%	    {2}{\draw[thepink,->,bend left]  (a) to (2);}
%	  \DEFAULT{}
%	  \node[circ,outer sep=2pt,draw opacity=0] at (0,9) {};
%	\end{tikzpicture}
%}
%
%\newcommand\parityNetD[1]{
%  \SWITCH{#1}
%  {3}{\begin{tikzpicture}[x=5mm,y=-4mm,octo]
%	  \node[pn] (a) at (0,2) {$\bot$};
%	  \node[pn] (2) at (1,7) {$\nx b\1$};
%	  \node[pn] (1) at (0,9) {$\nx a\1$};
%	  %
%	  \draw (0,0) -- (1,2) -- (a) -- (-1,2) -- (0,0);
%	  \draw (1) -- (2) -- (2,9) -- (1);
%	  %
%	  \draw[thepink,->,bend left] (a) to (2);
%	\end{tikzpicture}
%  }
%  \DEFAULT{
%	\begin{tikzpicture}[x=5mm,y=-4mm,octo]
%	  \node[pn] (a) at (0,2) {$\bot$};
%	  \node[pn] (1) at (0,5) {$\nx a\1$};
%	  \node[pn] (2) at (1,7) {$\nx b\1$};
%	  %
%	  \draw (0,0) -- (1,2) -- (a) -- (-1,2) -- (0,0);
%	  \draw (1) -- (2) -- (-1,7) -- (1);
%	  %
%	  \SWITCH{#1}
%	    {1}{\draw[thepink,->,bend left]  (a) to (1);}
%	    {2}{\draw[thepink,->,bend left] (a) to (2);}
%	  \DEFAULT{}
%	  \node[circ,outer sep=2pt,draw opacity=0] at (0,9) {};
%	\end{tikzpicture}
%  }
%}
%\newcommand\parityNetE{
%  \begin{tikzpicture}[x=1mm,y=-7mm]
%	\node[pn] (a1) at ( 0,0) {$a$};
%	\node[pn] (a2) at (10,0) {$v_0$};
%	\node[pn] (a3) at (20,0) {$v_1$};
%	\node[pn] (an) at (40,0) {$v_{2n}$};
%	\node[pn] (b) at (50,0) {$b$};
%	\node at (30,.5) {$\dots$};
%	%
%	\draw ($(a1)+(2,0)$)--(a2)--(a3)--($(a3)+(4,0)$) ($(an)+(-5,0)$)--(an)--(b);
%	%
%	\foreach \a in {a1,a2,a3,an,b} {
%	  \draw (\a)--+(4,1)--+(-4,1)--(\a);
%	}
%  \end{tikzpicture}
%}





\newcommand\parityNetC[1]{
	\begin{tikzpicture}[x=5mm,y=-4mm,octo]
	  \node[circ,  outer sep=2pt] (1) at (-1,2) {};
	  \node[circ,  outer sep=2pt] (2) at ( 1,2) {};
	  \node[bullet,outer sep=2pt] (a) at ( 0,5) {};
	  %
	  \node[anchor=north east] at (1) {$a\vphantom b$};
	  \node[anchor=north west] at (2) {$b$};
	  %
	  \draw (0,0) -- (1) -- (2) -- (0,0);
	  \draw (a) -- (1,7) -- (-1,7) -- (a);
	  %
	  \SWITCH{#1}
	    {1}{\draw[thepink,->,bend right] (a) to (1);}
	    {2}{\draw[thepink,->,bend left]  (a) to (2);}
	  \DEFAULT{}
	\end{tikzpicture}
}

\newcommand\parityNetD[1]{
  \SWITCH{#1}
  {3}{\begin{tikzpicture}[x=5mm,y=-4mm,octo]
	  \node[bullet,outer sep=2pt] (a) at (0,2) {};
	  \node[circ,  outer sep=2pt] (2) at (1,7) {};
	  \node[circ,  outer sep=2pt] (1) at (0,9) {};
	  %
	  \node[anchor=south east] at (1) {$a$};
	  \node[anchor=south east] at (2) {$b$};
	  %
	  \draw (0,0) -- (1,2) -- (a) -- (-1,2) -- (0,0);
	  \draw (1) -- (2) -- (2,9) -- (1);
	  %
	  \draw[thepink,->,bend left] (a) to (2);
	\end{tikzpicture}
  }
  \DEFAULT{
	\begin{tikzpicture}[x=5mm,y=-4mm,octo]
	  \node[bullet,outer sep=2pt] (a) at (0,2) {};
	  \node[circ,  outer sep=2pt] (1) at (0,5) {};
	  \node[circ,  outer sep=2pt] (2) at (1,7) {};
	  %
	  \node[anchor=south east] at (1) {$a$};
	  \node[anchor=north west] at (2) {$b$};
	  %
	  \draw (0,0) -- (1,2) -- (a) -- (-1,2) -- (0,0);
	  \draw (1) -- (2) -- (-1,7) -- (1);
	  %
	  \SWITCH{#1}
	    {1}{\draw[thepink,->,bend left]  (a) to (1);}
	    {2}{\draw[thepink,->,bend left] (a) to (2);}
	  \DEFAULT{}
	  \node[circ,outer sep=2pt,draw opacity=0] at (0,9) {};
	\end{tikzpicture}
  }
}

