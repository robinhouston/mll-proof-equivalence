\documentclass[conference]{IEEEtran}


\usepackage{amsmath,amssymb,amsthm}
\usepackage{mllequiv,willemtools}

\usepackage{pgffor}

\graphicspath{{../notes/}}

\makeTheoremDefs


\begin{document}


\[
	\lesssim \gtrsim
\[

\end{document}



\begin{lemma}
For a proof net $(\Gamma,\links)$, if the linking $\links'$ agrees with $\links$ on all values except $\links'(a)=v$, then $\links'$ is a proof net if and only if $v$ is in the empire of $a$ in $(\Gamma,\links)$.
%
%If $\links\perm*\links'$ by rewiring the link $\links(a)=v$ to $\links'(a)=w$, then $w$ is in the empire of $a$ in the proof net $\links$.
\end{lemma}

\begin{proof}
The empire of $a$ corresponds to the largest subproof $\Sigma$ in any $\Pi\toNet\links$ with as conclusion the introduction rule of $\named a\bot$.
%
In one direction, by Definition~\ref{def:proofs to nets} $a$ may link anywhere in a translation of $\Sigma$ to a net; if $\named v\1$ is in $\Sigma$, then $\links'$ is a net.
%
In the other direction, if $\links\perm*\links'$, then for any switching of $(\Gamma,\links')$ there is a path from $\links(a)=w$ to 

%%Let $\sigma$ be the empire of $a$ in $(\Gamma,\links)$.
%%
%In one direction, if $v$ is in $\sigma$, then since $\sigma$ is a subnet, for any switching of $\sigma$ there is a path from $\links(a)$ to $v$ not passing through $a$.
%%
%Then if $\sigma'\perm*\sigma$ by 
%
% $\sigma$ minus the link $\edge a{\links(a)}$ is a subnet, 
%
%By definition, $\sigma$ is the largest subnet with $a$ as a port.
%%
%Then there is a proof $\Sigma\tonet\sigma$ with as conclusion the introduction rule for $\named a\bot$, which 1) is a subproof of some sequent proof $\Pi\tonet\links$, and 2) is the largest subproof of $\named a\bot$ among all $\Pi\tonet\links$.
%
%
%In one direction, if $\links'$ is a proof net, then $\links\perm*\links'$, and there is a $\Pi'\tonet\links'$ such that $\Pi\perm\Pi'$ by permuting the introduction rule for $\named a\bot$.
%%
%But since permuting this rule in $\Pi$ cannot give it a larger subnet than $\Sigma$ (by 2 above), $\named v\1$ must be in $\Sigma$, and $v$ in $\sigma$.
%
%By Definition~\ref{def:proofs to nets} 

\end{proof}



\end{document}
%
%
%\begin{frame}
%	\octorollA1
%\end{frame}
%
%\begin{frame}
%	\octorollB1
%\end{frame}
%
%\begin{frame}
%	\octorollB2
%\end{frame}
%
%\begin{frame}
%	\octorollB3
%\end{frame}
%
%\begin{frame}
%	\octorollC
%\end{frame}
%
%\begin{frame}
%	\octorollD1
%\end{frame}
%
%\begin{frame}
%	\octorollE1
%\end{frame}
%
%\begin{frame}
%	\octorollE2
%\end{frame}
%
%\begin{frame}
%	\octorollE3
%\end{frame}
%
%\begin{frame}
%	\octorollE4
%\end{frame}
%
%\begin{frame}
%	\octorollE5
%\end{frame}
%
%\begin{frame}
%	\octorollE6
%\end{frame}
%
%\begin{frame}
%	\octorollE7
%\end{frame}

\begin{frame}
	\octorollD2
\end{frame}

\begin{frame}
	\octorollA2
\end{frame}

\end{document}




%\newcommand\octorollC{
%  \begin{tikzpicture}[x=5mm,y=-5mm,octo]
%	\node[circ] (a1a) at (0,1.5) {}; \node[big1] (a1) at (-2,0) {}; \node[big1] (a2) at (0,0) {}; \node[big1] (a3) at (2,0) {};
%	\draw (-9  ,4) node[bullet] (d1a) {} -- (-8  ,4) node[big_] (d1) {};
%	\draw (-6.5,4) node[bullet] (d2a) {} -- (-5.5,4) node[big_] (d2) {};
%	\draw (-4  ,4) node[bullet] (d3a) {} -- (-3  ,4) node[big_] (d3) {};
%	\draw[rounded corners] (-10,3) rectangle (-2,5);
%	\draw (-2,4)--(-1,4) node[bullet] (d) {};
%	\draw ( 3  ,4) node[bullet] (e1a) {} -- ( 4  ,4) node[big_] (e1) {};
%	\draw ( 5.5,4) node[bullet] (e2a) {} -- ( 6.5,4) node[big_] (e2) {};
%	\draw ( 8  ,4) node[bullet] (e3a) {} -- ( 9  ,4) node[big_] (e3) {};
%	\draw[rounded corners] (10,3) rectangle (2,5);
%	\draw (2,4)--(1,4) node[bullet] (e) {};
%	\node[circ] (v1a) at (-4,8); \node[big1] (v1b) at (-3,8) {}; \node[big1] (v1c) at (-1.5,8) {};
%	\node[big1] (v2) at  ( 2.5,8) {}; \node[big1] (v2) at ( 2.5,8) {};
%	\draw[rounded corners] (-4.5,7) rectangle (-0.5,9);
%	\draw[rounded corners] ( 3.5,7) rectangle ( 0.5,9);
%	\draw (-0.5,8)--(0.5,8);
%	\node[circ] (v) at (0,6) {};
%	%
%	\begin{scope}[thepink,->]
%		\draw[bend left=15] (e) to (v);\draw[bend right=15] (d) to (v);
%		\draw[big>,bend right] (d1) to (v1); \draw[big>,bend right=20] (d2) to (v1); \draw[big>, bend right=20] (d3) to (v2);
%		\draw[bend right=15] (e1a) to (a1a); \draw[bend right=15] (e2a) to (a1a); \draw[bend right=15] (e3a) to (a1a); 
%		\draw[big>,bend right=20] (e1) to (a1);  \draw[big>,bend right=20] (e2) to (a2);  \draw[big>, bend right=20] (e3) to (a3);
%	\end{scope}
%  \end{tikzpicture}
%}

\[
	\big(\bigparr_{k\times c(v_i)}\big(\bigparr_{3i+2}1 \tn \bigparr_{3(n-i)+3}1\big) \big)\parr 1
\]
\[
	\big(\bigotimes_{k\times w(e)}\big(\bigotimes_{3i+2}\bot\parr\bigotimes_{3(j-i)+1}\bot\parr\bigotimes_{3(n-j)+3}\bot\big)\big)\tn\bot
\]


\[
	C = \bigparr(1^{3i+2}) \tn \bigparr(1^{3(n-i)+3})
\]

\[
	\big(\bigparr C^{k\times c(v_i)} \big)\parr 1
\]

\[
	W = \bigotimes(\bot^{3i+2})\parr\bigotimes(\bot^{3(j-i)+1})\parr\bigotimes(\bot^{3(n-j)+3}
\]

\[
	\big(\bigotimes W^{k\times w(e)}\big)\tn\bot
\]
