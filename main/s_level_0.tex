


%NOTE: need 'octopus' terminology here

%NOTE: will we need to count arms often enough to make it worthwile to write $|\Gamma|_\bot$ for the number of $\bot$-occurrences in a sequent $\Gamma$?


\section{Equivalence in the absence of $~\protect\parr$}





Let a \emph{1-alternation} sequent be one over formulae of the form $1$ or $\bot\tn\ldots\tn\bot$, where the number of $\bot$-subformulae is at least 2.
%
Such a sequent is inhabited exactly when the number of formulae in the sequent is one greater than the total number of $\bot$-subformulae it contains.
%
An inhabited 1-alternation sequent with only one tensor-formula, i.e.\ a sequent of the form $\1,\ldots,\1,\bot\tn\ldots\tn\bot$ with $n$ $\bot$-subformulae and $n$ $1$-subformulae, will admit $n!$ different proof nets, each with $n$ links.
%
Since no link can re-attach, its equivalence classes are singletons.



\begin{proposition}
\label{prop:level0 max binary}
%
For a 1-alternation sequent with at least two tensor-formulae there are at most two equivalence classes of proof nets.
%
\end{proposition}


%
%\begin{figure}[p]
%\[
%\begin{array}{ccc}
%	\netA 1213 & \perm	& \netA 3213 \\ \\[-6pt] \vperm && \vperm \\ \\[-6pt]
%	\netA 1223 &		& \netA 3212 \\ \\[-6pt] \vperm && \vperm \\ \\[-6pt]
%	\netA 1323 &		& \netA 3112 \\ \\[-6pt] \vperm && \vperm \\ \\[-6pt]
%	\netA 1321 &		& \netA 3132 \\ \\[-6pt] \vperm && \vperm \\ \\[-6pt]
%	\netA 2321 &		& \netA 2132 \\ \\[-6pt] \vperm && \vperm \\ \\[-6pt]
%	\netA 2331 & \perm	& \netA 2131
%\end{array}
%\qquad
%\begin{array}{ccc}
%	\netA 1312 & \perm	& \netA 1332 \\ \\[-6pt] \vperm && \vperm \\ \\[-6pt]
%	\netA 2312 &		& \netA 1232 \\ \\[-6pt] \vperm && \vperm \\ \\[-6pt]
%	\netA 2313 &		& \netA 1231 \\ \\[-6pt] \vperm && \vperm \\ \\[-6pt]
%	\netA 2113 &		& \netA 3231 \\ \\[-6pt] \vperm && \vperm \\ \\[-6pt]
%	\netA 2123 &		& \netA 3221 \\ \\[-6pt] \vperm && \vperm \\ \\[-6pt]
%	\netA 3123 & \perm	& \netA 3121
%\end{array}
%\]
%\caption{The two equivalence classes of nets for $\1,\1,\1,\bot\tn\bot,\bot\tn,\bot$}
%\label{fig:2x12 nets}
%\end{figure}



\begin{proof}
%
It will be shown by induction on the number of $\bot$-formulae in $\Gamma$ that every proof net for $\Gamma$ belongs to one of two equivalence classes.
%
For the base case, the smallest inhabited sequent with two tensor-formulae is the following.
\[
	\1,\1,\1,\bot\tn\bot,\bot\tn,\bot
\]
It has two equivalence classes, of 12 proof nets each.
%
(Apart from listing these exhaustively, this can also be shown by using the proof of the inductive step, below, to reduce the base case to that of the sequent $\1,\1,\bot\tn\bot$, which has two singleton equivalence classes.)

%, displayed in Figure~\ref{fig:2x12 nets}.



For the inductive step, let $\Gamma$ be the following sequent.
\[
	\Delta,A\tn\named a\bot,\named z\1
\]
There are two cases: 1) where $A$ is a tensor-formula, and 2) where $A$ is $\bot$ and where, for the induction hypothesis to apply, $\Delta$ contains at least two tensor-formulae.
%
For both cases, it will be shown that any net $\links$ for $\Gamma$ is equivalent to a net $\links'$ where $\named a\bot$ connects to $\named z1$, and is the only link to do so.
%
This reduces equivalence on $\Gamma$ to equivalence on $\Delta,A$ in case 1, and on $\Delta$ in case 2, so that the induction hypothesis applies.


In constructing $\links'$, since the proof net must be connected, there is a path from $a$ to $z$.
%
We will consider only the jumps on this path, not any other edges.
%
There are four cases.

\begin{enumerate}[label=\roman*.]
	\item
The path consists of exactly the jump $\edge az$.
%
Let $\edge by$ be a jump from a $\named b\bot$ in $A$; then $\links'$ is obtained from $\links$ by changing: $\links'(e)=y$  for every $\named e\bot$ such that $\links(e)=z$ and $e\neq a$.


	\item
The path starts with the jump $\edge ax$ (and $x\neq z$).
%
Let $\edge by$ be a jump from a $\named b\bot$ in $A$, and let the path end with the jumps $\edge wc$ and $\edge dz$, where $\named c\bot$ and $\named d\bot$ are in the same tensor-formula $B$.
%
Then $\links'$ is obtained from $\links$ by changing: $\links'(a)=z$, and $\links'(e)=w$ for every $\named e\bot$ such that $\links(e)=z$ and $e\neq a$, including $d$.
%
These changes are illustrated as permutations below.
%
\[
	\netB xywz \perm \netB zywz \perm \netB zywy
\]



	\item
The path consists of exactly one jump $\edge bz$ from a $\named b\bot$ in $A$ (and $b\neq a$).
%
Let $\links(a)=y$.
%
Choose a jump $\edge cw$ from $\named c\bot$ in $A$ such that there is another $\edge dw$ from $\named d\bot$ in a formula $B$ (not excluding the possibilities $c=b$ and $c=a$).
%
Let $\edge ex$ be a further jump from $\named e\bot$ in $B$.
%
Then $\links'$ is obtained from $\links$ by changing: $\links'(a)=z$, $\links'(d)=x$, $\links'(e)=y$, and $\links'(f)=x$ for each $f$ (including $b$) such that $\links(f)=z$ and $f\neq a$.
%
These changes are exhibited as a series of permutations below, from top left to bottom left (note that the jump from $d$ moves twice).
%
\[
\begin{array}{ccccc}
	\netC yzwx & \perm & \netC yzzx & \perm & \netC yxzx
	\\ &&&& \vperm \\
	\netC zxxy & \perm & \netC zxzy & \perm & \netC yxzy
\end{array}
\]



	\item
The path starts with a jump $\edge bx$ from a $\named b\bot$ in $A$ (and $b\neq a$, $x\neq z$).
%
Let the path end with the jumps $\edge wc$ and $\edge dz$, and let $\links(a)=y$.
%
Then $\links'$ is obtained from $\links$ by changing: $\links'(c)=y$, $\links'(a)=z$, and $\links'(e)=w$ for every $\named e\bot$ such that $e\neq a$ and $\links(e)=z$, including $d$.
%
This is illustrated below.
\end{enumerate}
%
\[
	\kern-1pt \netB yxwz \perm \netB yxyz \perm \netB zxyz \perm \netB zxyw \kern-1pt 
\]
%
\end{proof}




\begin{proposition}
\label{prop:level0 may-connect path}
%
For a proof net for a 1-alternation sequent containing a link $\edge{\named a\bot}{\named b\1}$ and a formula $\named c\1$, the following are equivalent.
%
\begin{itemize}
	\item
The edge $\edge ab$ can be permuted to $\edge ac$.
	\item
There is a path from $b$ to $c$ not passing through $a$.
	\item
The path from $a$ to $c$ starts with the jump $a-b$.
\end{itemize} 
\end{proposition}


If a link $\edge ab$ may be reconnected as $\edge ac$ it is said that $a$ \emph{may connect to} $c$. 
%
By the above proposition, it is immediate that if $a$ and $b$ may both connect to $c$, then after actually reconnecting $\edge ac$, still $b$ may connect to $c$.



Consider the following naming scheme for the units in a 1-alternation sequent $\Gamma$ with tensor-formulae $A_1,\dotsc,A_n$.
%
\begin{itemize}

	\item
One $\1$ in $\Gamma$ is named $*$, and the remaining ones with the numbers $n+1,\dotsc,m$.

	\item
A $\bot$-formula in $A_i$ is named by a pair $(i,k)$, where $k=*$ for the first $\bot$-formula in each $A_i$, and for the remaining $\bot$-formulae in all $A_i$, each $k$ is a distinct number in $n+1,\dotsc,m$.

\end{itemize}
%
The naming scheme suggests a linking for $\Gamma$, defined by $\links(i,k)=k$; i.e\ the first $\bot$ in each tensor-formula connects to $\named *\1$, while other $\bot$-subformulae connect uniquely to the remaining $\1$-subformulae.



A net for $\Gamma$ is interpreted as a combinatorial permutation (an automorphism on $\{1,\dotsc,m\}$) as follows.
%
\begin{definition}
\label{def:combinatorial permutation}
To a proof net $\links$ for a 1-alternation sequent $\Gamma$ named as above, associate the \emph{permutation} $\prm:\{1,\dotsc,m\}\to\{1,\dotsc,m\}$ given by:
\[
	\prm(k) = 
	\begin{cases}
		i				& \text{ if $(i,k)$ may connect to $*$; and}
	\\	\links(i,k)		& \text{ otherwise.}
	\end{cases}
\]
The \emph{parity} of $\links$ is the parity of its permutation.
\end{definition}


To see that $\prm$ is injective, consider the following.
\begin{itemize}
	\item The domains of $i$ and $\links(i,k)$, respectively $1,\dotsc,n$ and $n+1,\dotsc,m$, are disjoint.
	\item Exactly one $\bot$-formula in each $A_i$ may connect to $*$ because of connectedness and acyclicity, since if a $\bot$-formula may connect to $*$ it has a path to $*$ (Proposition~\ref{prop:level0 may-connect path}).
	\item If two $\bot$-formulae have the same target, which means they are in different tensor-formulae, at least one may connect to $*$ via the other tensor-formula, which must have a path to $*$ by the above.
\end{itemize}



\begin{proposition}
\label{prop:level0 min binary}
A permutation on a net $\links$ preserves its parity. 
\end{proposition}


\begin{proof}
Let $\links$ be a net for $\Gamma$, with $\Gamma$ named as above, and let the link $\edge{(i,k)}x$ in $\links$ re-attach as $\edge{(i,k)}y$, forming $\links'$.
%
There are two cases, depending on whether $(i,k)$ may connect to $*$.
%
If so, using Proposition~\ref{prop:level0 may-connect path}, the re-wiring preserves which $\bot$-formulae may connect to $*$, since for any path to $*$ via $\edge{(i,k)}x$ in $\links$ there is a path to $*$ via $\edge{(i,k)}y$.
%
Then the permutation of $\links'$ is that of $\links$.


If $(i,k)$ may not connect to $*$, let the path from $x$ to $y$ run via the following $\bot$- and $\1$-vertices.
\[
	x=x_1, (i_1,j_1), (i_1,k_1), x_2, (i_2,j_2), \dotsc, (i_n,k_n), x_{n+1}=y 	
\]
Note that the $\bot$-formulae $(i_a,j_a)$ may connect to $*$.
%
On the relevant domain, this gives the following permutation for $\links$.
\[
\left(\begin{array}{ccccccc}
	j_1 & \dotso & j_n &  k  & k_1 & \dotso & k_n \\
	i_1 & \dotso & i_n & x_1 & x_2 & \dotso & x_{n+1}
\end{array}\right)
\]
In $\links'$, since $\links'(i,k)=y$, the $\bot$-formulae that may connect to $*$ are the $(i_a,k_a)$.
%
The permutation $\prm[\links']$ is the following.
\[
\left(\begin{array}{ccccccc}
	j_1 & \dotso & j_n &    k    & k_1 & \dotso & k_n \\
	x_1 & \dotso & x_n & x_{n+1} & i_1 & \dotso & i_n
\end{array}\right)
\]
The parity of both permutations is the same if and only if the relative permutation, below, is even.
\[
\left(\begin{array}{ccccccc}
	i_1 & \dotso & i_n & x_1     & x_2 & \dotso & x_{n+1} \\
	x_1 & \dotso & x_n & x_{n+1} & i_1 & \dotso & i_n
\end{array}\right)
\]
This is the case, as it is obtained by the exchange of $x_a$ and $i_a$ for each $a\leq n$, and subsequently the exchange of $x_{n+1}$ and each $i_a$ in turn.
%
\end{proof}




\begin{theorem}
\MLL\ proof equivalence in the absence of ~$\parr$ is linear-time decidable.
\end{theorem}

\begin{proof}
For a sequent with 1 tensor-formula, the problem is reduced to syntactic equality.
%
For a sequent with 2 or more tensor-formulae, by Propositions~\ref{prop:level0 max binary} and~\ref{prop:level0 min binary} the equivalence of two nets is determined by their relative parity.
%
Following Definition~\ref{def:combinatorial permutation} the parity of a net can be read off in a single traversal of the net.
%
This yields a linear-time algorithm.
%
\end{proof}











