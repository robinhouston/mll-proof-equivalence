

\newcommand\1{1}
\newcommand\prm[1][\links]{\mathit p_{#1}}


%NOTE: need 'octopus' terminology here

%NOTE: will we need to count arms often enough to make it worthwile to write $|\Gamma|_\bot$ for the number of $\bot$-occurrences in a sequent $\Gamma$?


\section{Equivalence in the absence of $~\protect\parr$}





Let a \emph{1-alternation} sequent be one over formulae of the form $1$ or $\bot\tn\ldots\tn\bot$, where the number of $\bot$-subformulae is at least 2.
%
Such a sequent is inhabited exactly when the number of formulae in the sequent is one greater than the total number of $\bot$-subformulae it contains.
%
An inhabited 1-alternation sequent with only one tensor-formula, i.e.\ a sequent of the form $\1,\ldots,\1,\bot\tn\ldots\tn\bot$ with $n$ $\bot$-subformulae and $n$ $1$-subformulae, will admit $n!$ different proof nets, each with $n$ links.
%
Since no link can re-attach, its equivalence classes are singletons.





\begin{proposition}
\label{lem:level0 max binary}
%
For a 1-alternation sequent with at least two tensor-formulae there are at most two equivalence classes of proof nets.
%
\end{proposition}



\begin{proof}
%
It will be shown by induction on the number of $\bot$-formulae in $\Gamma$ that every proof net for $\Gamma$ belongs to one of two equivalence classes.
%
For the base case, the smallest inhabited sequent with two tensor-formulae is the following.
\[
	\1,\1,\1,\bot\tn\bot,\bot\tn,\bot
\]
It has two equivalence classes of 12 proof nets each.



For the inductive step, let $\Gamma$ be the following sequent.
\[
	\Delta,A\tn\named a\bot,\named x\1
\]
There are two cases: 1) where $A$ is a tensor-formula, and 2) where $A$ is $\bot$ and where, for the induction hypothesis to apply, $\Delta$ contains at least two tensor-formulae.
%
For both cases, it will be shown that any net $\links$ for $\Gamma$ is equivalent to a net $\links'$ where $\named a\bot$ connects to $\named x1$, and is the only link to do so.
%
This reduces equivalence on $\Gamma$ to equivalence on $\Delta,A$ in case 1, and on $\Delta$ in case 2, so that the induction hypothesis applies.


Let $\named a\bot$ connect to $\named y\1$ in $\Delta$, and let $\named c\bot$ be a subformula of $A$, which means that for case 2, $A=\named c\bot$.
%
Then $\links'$ is obtained by adjusting $\links$ as follows.
\begin{itemize}

	\item
Let $c-z$, i.e.\ $\named z\1$ is the target of the jump from $\named c\bot$.	
%
Ensure that $\named z\1$ is the only target shared between jumps in $A$ and in $\Delta$, by moving any other such jump from $\Delta$ to $\named z\1$.

	\item
If there are multiple links connecting to $\named x\1$, select one $b-x$ for some $\named b\bot$ in a tensor-formula $B$.
%
Re-attach the others to the target of another jump out of $B$, of which there must be at least one.

	\item
Since $A$ is only connected via $\named z\1$, there is a jump $d-z$ connecting $B$ to $A$ (though $\named d\bot$ is not necessarily a subformula of $B$).
%
Re-attach $\named d\bot$ to $\named y\1$, then $\named a\bot$ to $\named x\1$, and $\named b\bot$ to $\named z\1$.

\end{itemize}

\end{proof}



Consider the following naming scheme for the units in a 1-alternation sequent $\Gamma$ with tensor-formulae $A_1,\dotsc,A_n$.
%
\begin{itemize}

	\item
The first $\bot$ in each $A_i$ is named $b(i)$, while the remaining $\bot$-subformulae in $\Gamma$ are named $b(n+1),\dotsc,b(m)$.

	\item
The first $\1$ in $\Gamma$ is named by the set $A=\{1,\dotsc,n\}$, the remaining ones are named $n+1,\dotsc,m$.

\end{itemize}
%
The naming scheme suggests a linking for $\Gamma$, where the first $\bot$-subformula of each tensor-formula connects to $\named A\1$, while other $\bot$-subformulae connect uniquely to the remaining $\1$-subformulae; i.e.\ $b(i)-A$ when $i\leq n$ and $b(i)-i$ otherwise.
%
A net for $\Gamma$ can be interpreted as a permutation (an automorphism on $\{1,\dotsc,m\}$) as follows.
%
\begin{definition}
To a proof net $\links$ for a 1-alternation sequent $\Gamma$ named as above, associate the \emph{permutation} $\prm:\{1,\dotsc,m\}\to\{1,\dotsc,m\}$ given by:
\[
	\prm(i) = 
	\begin{cases}
		j 				& \text{ if $b(i)$ may connect to $A$ and $\named{b(i)}\bot$ is a subformula of $t(j)$}
	\\	\links(b(i))	& \text{ otherwise.}
	\end{cases}
\]
The \emph{parity} of $\links$ is the parity of its permutation.
\end{definition}



\begin{proposition}
\label{lem:level0 min binary}
Re-attaching a jump in a net $\links$ preserves its parity. 
\end{proposition}


\begin{proof}
TO DO
\end{proof}















