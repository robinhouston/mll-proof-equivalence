


%NOTE: need 'octopus' terminology here

%NOTE: will we need to count arms often enough to make it worthwile to write $|\Gamma|_\bot$ for the number of $\bot$-occurrences in a sequent $\Gamma$?


\section{Equivalence in the absence of $~\protect\parr$}





Let a \emph{1-alternation} sequent be one over formulae of the form $1$ or $\bot\tn\ldots\tn\bot$, where the number of $\bot$-subformulae is at least 2.
%
Such a sequent is inhabited exactly when the number of formulae in the sequent is one greater than the total number of $\bot$-subformulae it contains.
%
An inhabited 1-alternation sequent with only one tensor-formula, i.e.\ a sequent of the form $\1,\ldots,\1,\bot\tn\ldots\tn\bot$ with $n$ $\bot$-subformulae and $n$ $1$-subformulae, will admit $n!$ different proof nets, each with $n$ links.
%
Since no link can re-attach, its equivalence classes are singletons.



\begin{proposition}
\label{prop:level0 max binary}
%
For a 1-alternation sequent with at least two tensor-formulae there are at most two equivalence classes of proof nets.
%
\end{proposition}



\begin{figure}[p]
\[
\begin{array}{ccc}
	\basenet 1213 & \perm & \basenet 3213 \\ \\[-6pt] \vperm && \vperm \\ \\[-6pt]
	\basenet 1223 &		  & \basenet 3212 \\ \\[-6pt] \vperm && \vperm \\ \\[-6pt]
	\basenet 1323 &		  & \basenet 3112 \\ \\[-6pt] \vperm && \vperm \\ \\[-6pt]
	\basenet 1321 &		  & \basenet 3132 \\ \\[-6pt] \vperm && \vperm \\ \\[-6pt]
	\basenet 2321 &		  & \basenet 2132 \\ \\[-6pt] \vperm && \vperm \\ \\[-6pt]
	\basenet 2331 & \perm & \basenet 2131
\end{array}
\qquad
\begin{array}{ccc}
	\basenet 1312 & \perm & \basenet 1332 \\ \\[-6pt] \vperm && \vperm \\ \\[-6pt]
	\basenet 2312 &		  & \basenet 1232 \\ \\[-6pt] \vperm && \vperm \\ \\[-6pt]
	\basenet 2313 &		  & \basenet 1231 \\ \\[-6pt] \vperm && \vperm \\ \\[-6pt]
	\basenet 2113 &		  & \basenet 3231 \\ \\[-6pt] \vperm && \vperm \\ \\[-6pt]
	\basenet 2123 &		  & \basenet 3221 \\ \\[-6pt] \vperm && \vperm \\ \\[-6pt]
	\basenet 3123 & \perm & \basenet 3121
\end{array}
\]
\caption{The two equivalence classes of nets for $\1,\1,\1,\bot\tn\bot,\bot\tn,\bot$}
\label{fig:2x12 nets}
\end{figure}



\begin{proof}
%
It will be shown by induction on the number of $\bot$-formulae in $\Gamma$ that every proof net for $\Gamma$ belongs to one of two equivalence classes.
%
For the base case, the smallest inhabited sequent with two tensor-formulae is the following.
\[
	\1,\1,\1,\bot\tn\bot,\bot\tn,\bot
\]
It has two equivalence classes of 12 proof nets each, displayed in Figure~\ref{fig:2x12 nets}.



For the inductive step, let $\Gamma$ be the following sequent.
\[
	\Delta,A\tn\named a\bot,\named x\1
\]
There are two cases: 1) where $A$ is a tensor-formula, and 2) where $A$ is $\bot$ and where, for the induction hypothesis to apply, $\Delta$ contains at least two tensor-formulae.
%
For both cases, it will be shown that any net $\links$ for $\Gamma$ is equivalent to a net $\links'$ where $\named a\bot$ connects to $\named x1$, and is the only link to do so.
%
This reduces equivalence on $\Gamma$ to equivalence on $\Delta,A$ in case 1, and on $\Delta$ in case 2, so that the induction hypothesis applies.


Let $\named a\bot$ connect to $\named y\1$ in $\Delta$.
%
For case 1, let $A$ be $A'\tn\named c\bot$; for case 2, let $A=\named c\bot$.
%
Then $\links'$ is obtained by adjusting $\links$ as follows.
\begin{itemize}

	\item
Let $\edge cz$, i.e.\ $\named z\1$ is the target of the jump from $\named c\bot$.	
%
Ensure that $\named z\1$ is the only target shared between jumps in $A$ and in $\Delta$, by moving any other such jump from $\Delta$ to $\named z\1$.

	\item
If there are multiple links connecting to $\named x\1$, select one $b-x$ for some $\named b\bot$ in a tensor-formula $B$.
%
Re-attach the others to the target of another jump out of $B$, of which there must be at least one.

	\item
Since $A$ is only connected via $\named z\1$, there is a link $\edge dz$ connecting $B$ to $A$ (though $\named d\bot$ is not necessarily a subformula of $B$).
%
Re-attach $\named d\bot$ to $\named y\1$, then $\named a\bot$ to $\named x\1$, and $\named b\bot$ to $\named z\1$.

\end{itemize}

\end{proof}




\begin{proposition}
\label{prop:level0 may-connect path}
%
In a proof net for a 1-alternation sequent a link $\edge{\named a\bot}{\named b\1}$ can be permuted to $\edge{\named a\bot}{\named c\1}$ if and only if there is a path in the net from $b$ to $c$ not passing through $a$.
%
\end{proposition}


If a link $\edge ab$ may be reconnected as $\edge ac$ it is said that $a$ \emph{may connect to} $c$. 
%
By the above proposition, it is immediate that if $a$ and $b$ may both connect to $c$, then after actually reconnecting $\edge ac$, still $b$ may connect to $c$.



Consider the following naming scheme for the units in a 1-alternation sequent $\Gamma$ with tensor-formulae $A_1,\dotsc,A_n$.
%
\begin{itemize}

	\item
The first $\1$ in $\Gamma$ is named $N$, the remaining ones are named with the numbers $n+1,\dotsc,m$.

	\item
A $\bot$-formula in $A_i$ is named by a pair $(i,k)$, where $k=N$ for the first $\bot$-formula in each $A_i$, and for the remaining $\bot$-formulae in all $A_i$, each $k$ is a distinct number in $n+1,\dotsc,m$.

\end{itemize}
%
The naming scheme suggests a linking for $\Gamma$, where $\edge{(i,k)}{k}$ for each $\bot$-formula; i.e\ the first $\bot$-subformula of each tensor-formula connects to $\named N\1$, while other $\bot$-subformulae connect uniquely to the remaining $\1$-subformulae.



A net for $\Gamma$ is interpreted as a combinatorial permutation (an automorphism on $\{1,\dotsc,m\}$) as follows.
%
\begin{definition}
To a proof net $\links$ for a 1-alternation sequent $\Gamma$ named as above, associate the \emph{permutation} $\prm:\{1,\dotsc,m\}\to\{1,\dotsc,m\}$ given by:
\[
	\prm(k) = 
	\begin{cases}
		i				& \text{ if $(i,k)$ may connect to $N$; otherwise,}
	\\	j				& \text{ where $\edge{(i,k)}j$.}
	\end{cases}
\]
The \emph{parity} of $\links$ is the parity of its permutation.
\end{definition}


To see that $\prm$ is injective, consider the following.
\begin{itemize}
	\item The domains of $i$ and $j$ are $1,\dotsc,n$ and $n+1,\dotsc,m$ respectively, and hence disjoint.
	\item Exactly one $\bot$-formula in each $A_i$ may connect to $N$ because of connectedness and acyclicity, since if a $\bot$-formula may connect to $N$ it has a path to $N$ (Proposition~\ref{prop:level0 may-connect path}).
	\item If two $\bot$-formulae have the same target, which means they are in different tensor-formulae, at least one may connect to $N$ via the other tensor-formula, which must have a path to $N$ by the above.
\end{itemize}



\begin{proposition}
\label{lem:level0 min binary}
Re-attaching a jump in a net $\links$ preserves its parity. 
\end{proposition}


\begin{proof}
Let $\links$ be a net for $\Gamma$, with $\Gamma$ named as above, and let the link $\edge{(i,k)}x$ in $\links$ re-attach as $\edge{(i,k)}y$, forming $\links'$.
%
There are two cases, depending on whether $(i,k)$ may connect to $N$.
%
If so, using Proposition~\ref{prop:level0 may-connect path}, the re-wiring preserves which $\bot$-formulae may connect to $N$, since for any path to $N$ via $\edge{(i,k)}x$ in $\links$ there is a path to $N$ via $\edge{(i,k)}y$.
%
Then the permutation of $\links'$ is that of $\links$.


If $(i,k)$ may not connect to $N$, let the path from $x$ to $y$ run via the following $\bot$- and $\1$-vertices.
\[
	x=x_1, (i_1,j_1), (i_1,k_1), x_2, (i_2,j_2), \dotsc, (i_n,k_n), x_{n+1}=y 	
\]
Note that the $\bot$-formulae $(i_a,j_a)$ may connect to $N$.
%
On the relevant domain, this gives the following permutation for $\links$.
\[
\left(\begin{array}{ccccccc}
	j_1 & \dotso & j_n &  k  & k_1 & \dotso & k_n \\
	i_1 & \dotso & i_n & x_1 & x_2 & \dotso & x_{n+1}
\end{array}\right)
\]
In $\links$, where $(i,k)$ connects to $y$, the $\bot$-formulae $(i_a,k_a)$ may connect to $N$, giving the following permutation.
\[
\left(\begin{array}{ccccccc}
	j_1 & \dotso & j_n &    k    & k_1 & \dotso & k_n \\
	x_1 & \dotso & x_n & x_{n+1} & i_1 & \dotso & i_n
\end{array}\right)
\]
The parity of both permutations is the same if and only if the relative permutation, below, is even.
\[
\left(\begin{array}{ccccccc}
	i_1 & \dotso & i_n & x_1     & x_2 & \dotso & x_{n+1} \\
	x_1 & \dotso & x_n & x_{n+1} & i_1 & \dotso & i_n
\end{array}\right)
\]
This is the case, as it is obtained by the exchange of $x_a$ and $i_a$ for each $a\leq n$, and subsequently the exchange of $x_{n+1}$ and each $i_a$ in turn.

\end{proof}















